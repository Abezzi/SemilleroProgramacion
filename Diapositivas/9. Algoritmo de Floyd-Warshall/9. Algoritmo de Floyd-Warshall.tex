\documentclass{beamer}
\usepackage[utf8]{inputenc}
\usepackage[spanish]{babel}
\usepackage{hyperref}
\usepackage{verbatim}
\usepackage{listings}
\usepackage{tikz}
\usetikzlibrary{arrows}

\setbeamercovered{invisible}
\usetheme{Frankfurt}
\usefonttheme{serif}

% Configurar los listings (Códigos)
\renewcommand{\lstlistingname}{Código}
\lstset{
	language=C++,               % Lenguaje
	basicstyle=\ttfamily\footnotesize,  % Tipo de fuente
	keywordstyle=\color{blue},  % Color de palabras clave
	stringstyle=\color{red},    % Color de strings
	commentstyle=\color{gray},  % Color de comentarios
	showstringspaces=false,     % No muestrar el _ cuando el string tiene espacios
	breaklines = true,          % Partir las líneas largas
	breakatwhitespace=true,	    % Partir las líneas en un espacio
	numbers=left,				% Numerar las líneas a la izq
	numberstyle=\tiny,			% Poner los números de las líneas pequeños
	numberblanklines=true,      % Numerar las líneas en blanco
	columns=fullflexible,       % No perder el formato al dejar los espacios
	keepspaces=true,   			% Dejar los espacios insertados
	frame=tb,					% Poner el recuadro
}

\AtBeginSection[]{%
  \begin{frame}<beamer>
    \frametitle{Contenido}
    \tableofcontents[sectionstyle=show/hide,subsectionstyle=hide/show/hide]
  \end{frame}
  \addtocounter{framenumber}{-1}% If you don't want them to affect the slide number
}

\title{Semillero de Programación}
\subtitle{Algoritmo de Floyd-Warshall}
\author{Ana Echavarría \and Juan Francisco Cardona}

\institute{Universidad EAFIT}
\date{12 de abril de 2013}

\begin{document}

\begin{frame}
	\titlepage
\end{frame}

\begin{frame}
	\frametitle{Contenido}
	\tableofcontents
\end{frame}

\section{Problemas semana anterior}
	\subsection{Problema 1 - Cut Ribbon}
	\begin{frame}
		\frametitle{Problema 1 - Cut Ribbon}
		Hay que hallar el mínimo número de cortes que hay que hacerle a una cinta de longitud $n$ para que queden únicamente pedazos de tamaño $a$, $b$ o $c$.\\
		Este problema es un caso del problema coin change que consiste en hallar el mínimo número de monedas/billetes que hay que usar para tener una cantidad $v$ de dinero.\\
	\end{frame}
	
	\begin{frame}
		\frametitle{Problema 1 - Cut Ribbon}
		Sea $f(n)$ el mínimo número de cortes que hay que hacerle a una cinta de tamaño $n$ para que queden solo cintas de tamaño $a$, $b$ o $c$.\\
		$f(0) = 0$ \\
		$
		f(i) =  \text{min } 
		\left\{
			\begin{aligned}
				& \infty\\
				& 1 + f(i-a) \text{\quad si } i \geq a\\
				& 1 + f(i-b) \text{\quad si } i \geq b\\
				& 1 + f(i-c) \text{\quad si } i \geq c\\
			\end{aligned}
		\right\}
		\text{ para } 
			\begin{aligned}
				& 1 \leq i \leq n\\
			\end{aligned}
		$
	\end{frame}
	
	\begin{frame}[fragile, allowframebreaks]
		\frametitle{Implementación}
		\begin{lstlisting}
			const int MAXN = 4005;
			const int INF = 1 << 30;
			int dp[MAXN];

			int main(){
			    int n, a, b, c;
			    while (cin >> n >> a >> b >> c){
			        dp[0] = 0;
			        for (int i = 1; i <= n; ++i){
			            dp[i] = INF;
			            if (i >= a) dp[i] = max(dp[i], 1 + dp[i-a]);
			            if (i >= b) dp[i] = max(dp[i], 1 + dp[i-b]);
			            if (i >= c) dp[i] = max(dp[i], 1 + dp[i-c]);
			        }
			        cout << dp[n] << endl;
			    }
			    return 0;
			}
		\end{lstlisting}
	\end{frame}
	
	\subsection{Problema 2 - Scuba diver}
	\begin{frame}
		\frametitle{Problema 2 - Scuba diver}
		Este problema es parecido al problema de la mochila pero con las siguientes dos modificaciones:
		\begin{itemize}
			\item El problema es de minimización por lo que no se restringe a una capacidad de máximo $j$ sino una capacidad de mínimo $j$.
			\item Se tienen dos ``mochilas'' para llenar y se deben cumplir las restricciones de ambas.
		\end{itemize}		
	\end{frame}
	
	\begin{frame}
		\frametitle{Problema 2 - Scuba diver}
		Sea $dp(i, j, k)$ el mínimo peso que se puede llevar si se usan los tanques $1 \ldots i$ y se quieren llevar mínimo $j$ litros de oxígeno y $k$ litros de nitrógeno.\\ \quad \\
		$dp(0, j, k) = \infty \text{\quad para } 1 \leq j \leq minOxy \text{ y } 1 \leq k \leq minNitro$ \\
		$dp(0, 0, 0) = 0$\\
		$dp(i, j, k) =  \text{min } 
		\left\{
			\begin{aligned}
				& dp(i - 1, j, k)  \\
			 	& dp(i - 1, j - oxy[i], k - nitro[i] )
			\end{aligned}
		\right\}
		\text{ para } 
			\begin{aligned}
				& 1 \leq i \leq n\\
				& 0 \leq j \leq minOxy\\
				& 0 \leq k \leq minNitro\\
			\end{aligned}
		$\\ \quad \\
		Nota: Si $j - oxy[i] < 0$ o $k - nitro[i] < 0$ se tomará el valor de 0 ya que se está cumpliendo con la restricción de la mínima cantidad de oxígeno / nitrógeno.
	\end{frame}
	
	\begin{frame}[fragile, allowframebreaks]
		\frametitle{Implementación}
		\begin{lstlisting}
		const int MAXN = 1005;
		const int MAXOXY = 25, MAXNITRO = 85;
		const int INF = 1 << 30;
		int oxy[MAXN], nitro[MAXN], weight[MAXN];
		int dp[MAXN][MAXOXY][MAXNITRO];

		int main(){
		   int cases; cin >> cases;
		   while (cases--){
		      int min_oxy, min_nitro;
		      cin >> min_oxy >> min_nitro;
		      int n; cin >> n;
		      for (int i = 1; i <= n; ++i)
		         cin >> oxy[i] >> nitro[i] >> weight[i];

		      for (int j = 0; j <= min_oxy; ++j){
		         for (int k = 0; k <= min_nitro; ++k){
		            dp[0][j][k] = INF;
		         }
		      }
		      dp[0][0][0] = 0;

		      for (int i = 1; i <= n; ++i){
		         for (int j = 0; j <= min_oxy; ++j){
		            for (int k = 0; k <= min_nitro; ++k){
		               int dont_take = dp[i-1][j][k];
		               int take = weight[i] +                               dp[i-1][max(0, j-oxy[i])][max(0, k-nitro[i])];
		               dp[i][j][k] = min(dont_take, take);
		            }
		         }
		      }
		      cout << dp[n][min_oxy][min_nitro] << endl;
		   }
		   return 0;
		}
		\end{lstlisting}
	\end{frame}
	
	\subsection{Problema 3 - Knapsack}
	\begin{frame}
		\frametitle{Problema 3 - Knapsack}
		Implementar directamente el algoritmo para el problema de la mochila mostrado en la sesión anterior
	\end{frame}
	
	\subsection{Problema 4 - History Grading}
	\begin{frame}[fragile]
		\frametitle{Problema 4 - History Grading}
		Hay que hallar la longitud de la subsecuencia común más larga entre la secuencia base y las demás secuencias.\\
		Hay que tener cuidado ya que la entrada \verb|s| no es la secuencia en sí sino que \verb|s[i]| es la posición del número \verb|i| en la secuencia.
	\end{frame}
	
	\begin{frame}[fragile, allowframebreaks]
		\frametitle{Implementación}
		\begin{lstlisting}
			const int MAXN = 25;
			int order[MAXN];    // El orden de las secuencias
			int pattern[MAXN];  // La secuencia base
			int query[MAXN];    // La secuencia por la que se pregunta
			int dp[MAXN][MAXN]; // La dp para LCS

			int main(){
			   int n; cin >> n;
			   for (int i = 1; i <= n; ++i){
			      cin >> order[i];
			      pattern[order[i]] = i;
			   }

			   while (cin >> order[1]){
			      query[order[1]] = 1;
			      for (int i = 2; i <= n; ++i){
			         cin >> order[i];
			         query[order[i]] = i;
			      }
			
			      for (int j = 0; j <= n; ++j) dp[0][j] = 0;
			      for (int i = 0; i <= n; ++i) dp[i][0] = 0;

			      for (int i = 1; i <= n; ++i){
			         for (int j = 1; j <= n; ++j){
			            dp[i][j] = max(dp[i-1][j], dp[i][j-1]);
			            if (pattern[i] == query[j]){
			               dp[i][j] = max(dp[i][j], 1 + dp[i-1][j-1]);
			            }
			         }
			      }
			      cout << dp[n][n] << endl;
			   }
			    return 0;
			}
		\end{lstlisting}
	\end{frame}

\section[APSP Problem]{All-Pair Shortest Paths Problem}
	\begin{frame}
		\frametitle{All-Pair Shortest Paths Problem (APSP)}
		\textcolor{blue}{\large Entrada}\\
		Un grafo con pesos $G = (V, E)$ \\ \quad \\
		\textcolor{blue}{\large Objetivo}\\
		Hallar la distancia más corta entre todos los pares de nodos o decir que hay un ciclo de peso negativo en el grafo.
	\end{frame}
	
	\begin{frame}
		\frametitle{All-Pair Shortest Paths Problem (APSP)}
		\begin{alertblock}{Preguntas}
			Con los elementos que se han enseñado en el semillero ¿se puede hallar el camino más corto entre todas las parejas de nodos?
		\end{alertblock}
		\pause
		\begin{exampleblock}{Respuesta}
			Sí, se pueden utilizar los algoritmo de Dijkstra / Bellman-Ford que halla la distancia más corta entre un nodo y todos los demás. Estos algoritmos se corren desde cada uno de los nodos y con eso se tiene la distancia más corta cualquier nodo y todos los demás.
		\end{exampleblock}
	\end{frame}

\section[Floyd-Warshall]{Algoritmo de Floyd-Warshall}
	\begin{frame}
		\frametitle{Algoritmo de Floyd-Warshall}
		El algoritmo de Floyd-Warshall es un algoritmo que usa programación dinámica para hallar la distancia más corta entre todos los pares de nodos de un grafo con pesos. \\
		Para propósitos del problema consideremos que el grafo tiene $n$ nodos numerados desde $1$ hasta $n$.
	\end{frame}
	
	\begin{frame}
		\frametitle{Algoritmo de Floyd-Warshall}
		Sea $d(i, j, k)$ la distancia más corta entre el nodo $i$ el nodo $j$ si como nodos intermedios solo se pueden usar los nodos $1 \ldots k$.
		
		\begin{center}
			\begin{tikzpicture}[->,>=stealth',shorten >=1pt,auto,node distance=2cm,
			                    thick,main node/.style={circle,draw,font=\sffamily\Large\bfseries}]

				\node[main node] (1) {1};
				\node[main node] (2) [above right of=1] {2};
				\node[main node] (3) [right of=2] {3};
				\node[main node] (4) [below right of=2] {4};
				\node[main node] (5) [below right of=3] {5};

				\path[every node/.style={font=\sffamily\small}]
				(1) edge node {2} (2)
				    edge node {-5} (4)
				(2) edge node {1} (3)
				(3) edge node {-2} (5)
				(4) edge node {5} (5);
			\end{tikzpicture}
		\end{center}
		
		\begin{itemize}
			\item ¿Cuál sería el valor de $f(1, 2, 0)$ ? \pause \quad 2
			\item ¿Cuál sería el valor de $f(1, 5, 2)$ ? \pause \quad $+\infty$
			\item ¿Cuál sería el valor de $f(1, 5, 3)$ ? \pause \quad $1$
			\item ¿Cuál sería el valor de $f(1, 5, 4)$ ? \pause \quad $0$
		\end{itemize}
	\end{frame}
	
	\begin{frame}
		\frametitle{Algoritmo de Floyd-Warshall}
		Sea $d(i, j, k)$ la distancia más corta entre el nodo $i$ el nodo $j$ si como nodos intermedios solo se pueden usar los nodos $1 \ldots k$. \\ \vfill
		$d(i, j, 0) =
		\left\{
			\begin{aligned}
				& 0       &\text{\quad si } i = j\\
				& C_{i,j} &\text{\quad si } (i, j) \in E\\
				& +\infty &\text{\quad en otro caso}
			\end{aligned}
		\right\}
		\text{ para } 
			\begin{aligned}
				& 1 \leq i \leq n\\
				& 1 \leq j \leq n\\
			\end{aligned}
		$\\ \vfill 
		Donde $C_{i, j}$ es el costo que tiene asociado la arista $(i, j)$
	\end{frame}
	
	\begin{frame}
		\frametitle{Algoritmo de Floyd-Warshall}
		Sea $d(i, j, k)$ la distancia más corta entre el nodo $i$ el nodo $j$ si como nodos intermedios solo se pueden usar los nodos $1 \ldots k$. \\ \vfill
		$d(i, j, k) = \text{ min }
		\left\{
			\begin{aligned}
				& d(i, j, k-1)                \text{\qquad\qquad (no usar nodo } k) \\
				& d(i, k, k-1) + d(k, j, k-1) \text{\quad(usar } k) \\
			\end{aligned}
		\right\}
		$\\ \vfill
		Para $ 1 \leq i, j, k \leq n $\\
	\end{frame}
	
	\begin{frame}[fragile]
		\frametitle{Implementación}
		\begin{lstlisting}
			for (int i = 1; i <= n; ++i) {
			   for (int j = 1; j <= n; ++j){
			      d[i][j][0] = INF;
			   }
			   d[i][i][0] = 0;
			}

			for (int edge = 0; edge < m; ++edge){
			   int u, v, c; cin >> u >> v >> c;
			   d[u][v][0] = min(d[u][v], c); //Por si hay un loop de peso > 0
			}
			
			for (int k = 1; k <= n; ++k){
			   for (int i = 1; i <= n; ++i){
			      for (int j = 1; j <= n; ++j){
			         d[i][j][k] = min(d[i][j][k-1],
			                          d[i][k][k-1] + d[k][j][k-1]);
			      }
			   }
			}
		\end{lstlisting}
	\end{frame}
	
	\begin{frame}
		\frametitle{Complejidad}
		\begin{alertblock}{Preguntas}
			\begin{itemize}
				\item ¿Cuál es la complejidad \textbf{en tiempo} del algoritmo de Floyd-Warshall?
				\item ¿Cuál es la complejidad \textbf{en memoria} del algoritmo de Floyd-Warshall?
			\end{itemize}
		\end{alertblock}
		\pause
		\begin{exampleblock}{Respuestas}
			\begin{itemize}
				\item $O(n^3)$
				\item $O(n^3)$
			\end{itemize}
		\end{exampleblock}
	\end{frame}
	
	\begin{frame}[fragile]
		\frametitle{Optimización del espacio}
		El espacio del algoritmo de Floyd-Warshall se puede reducir a $O(n^2)$ (no almacenar la tercera dimensión) si se tiene en cuenta lo siguiente:
		\begin{itemize}
			\item si \verb|d[i][j][k] = d[i][j][k-1]| no importa que no se guarden los dos en el mismo lugar ya que son el mismo valor
			\item para \verb|d[i][j][k]| solo se usan los valores de \verb|d[i][k][k-1]| y \verb|d[k][j][k-1]|.\\Se puede ver que \verb|d[i][k][k-1]| = \verb|d[i][k][k]| y que \verb|d[k][j][k-1]| = \verb|d[k][j][k]| por lo que ninguno de los valores se sobreescribe y el algoritmo sigue siendo correcto.
		\end{itemize}
	\end{frame}
	
	\begin{frame}[fragile]
		\frametitle{Implementación con optimización del espacio}
		\begin{lstlisting}
			for (int i = 1; i <= n; ++i) {
			   for (int j = 1; j <= n; ++j){
			      d[i][j] = INF;
			   }
			   d[i][i] = 0;
			}

			for (int edge = 0; edge < m; ++edge){
			   int u, v, c; cin >> u >> v >> c;
			   d[u][v] = min(d[u][v], c); // Por si hay un loop de peso > 0
			}
			
			for (int k = 1; k <= n; ++k){
			   for (int i = 1; i <= n; ++i){
			      for (int j = 1; j <= n; ++j){
			         d[i][j] = min(d[i][j], d[i][k] + d[k][j]);
			      }
			   }
			}
		\end{lstlisting}
	\end{frame}
	
	\begin{frame}
		\frametitle{Detectando un ciclo de peso negativo}
		El grafo tiene un \textbf{ciclo de peso negativo}, si y solo si luego de la ejecución del algoritmo de Floyd-Warshall \textbf{algún elemento de la diagonal es negativo}. \\
	\end{frame}
	
	\begin{frame}[fragile]
		\frametitle{Demostración}
		$\rightarrow$ \\
		Supongamos que hay un ciclo de peso negativo que empieza en el nodo $i$. Sea $j$ el nodo más grande que pertenece a ese ciclo. En la ejecución del algoritmo cuando $k = j$ \\
		\verb|d[i][i] = min(d[i][i], d[i][j] + d[j][i])|\\
		Pero \verb|d[i][j] + d[j][i]| es el costo del ciclo ($< 0$) ya que el mayor nodo del ciclo es el nodo $k$ luego \verb|d[i][i]| $< 0$.\\
		\vfill 
		$\leftarrow$ \\
		Supongamos que \verb|d[i][i]| $< 0$ para algún $i$ luego de la ejecución del algoritmo. Claramente hay un ciclo de peso negativo que contiene el nodo $i$.
	\end{frame}

\section[Aplicaciones]{Otras aplicaciones del algoritmo de Floyd-Warshall}
	\begin{frame}
		\frametitle{Clausura transitiva}
		\textcolor{blue}{\large Entrada}\\
		Un grafo cualquiera $G = (V, E)$ \\ \quad \\
		\textcolor{blue}{\large Objetivo}\\
		Para cada pareja de nodos $(i, j)$ , hallar si existe un camino desde $i$ hasta $j$.
	\end{frame}
	
	\begin{frame}
		\frametitle{Clausura transitiva}
		\textcolor{blue}{\large Idea}\\
		Hay un camino desde el nodo $i$ hasta el nodo $j$ si y solo si pasa al menos una de las siguientes:
		\begin{itemize}
			\item El nodo $i$ es igual al nodo $j$ (caso base).
			\item Hay una arista del nodo $i$ al nodo $j$ (caso base).
			\item Existe un nodo $k$ tal que haya un camino del nodo $i$ al nodo $k$ y del nodo $k$ al nodo $j$ (caso recursivo).
		\end{itemize}
		\vfill
		{\Large Pensar en si hay un camino de $i$ hasta $j$ que como aristas intermedias las aristas $1 \ldots k$}
	\end{frame}
	
	\begin{frame}[fragile]
		\frametitle{Implementación}
		\begin{lstlisting}
			for (int i = 1; i <= n; ++i) {
			   for (int j = 1; j <= n; ++j){
			      d[i][j] = false;
			   }
			   d[i][i] = true;
			}

			for (int edge = 0; edge < m; ++edge){
			   int u, v; cin >> u >> v;
			   d[u][v] = true;
			}
			
			for (int k = 1; k <= n; ++k){
			   for (int i = 1; i <= n; ++i){
			      for (int j = 1; j <= n; ++j){
			         d[i][j] = d[i][j] or (d[i][k] and d[k][j]);
			      }
			   }
			}
		\end{lstlisting}
	\end{frame}
	
	\begin{frame}
		\frametitle{Minimax}
		\textcolor{blue}{\large Entrada}\\
		Un grafo con pesos $G = (V, E)$ \\ \quad \\
		\textcolor{blue}{\large Objetivo}\\
		Para cada pareja de nodos $(i, j)$ , hallar el camino de $i$ hasta $j$ donde la arista más grande del camino sea lo más pequeña posible.\\ \quad \\
		\textcolor{blue}{\large Ejemplos}\\
		Que el peaje más caro sea lo más barato posible.\\
		Que la autopista más larga sea lo más corta posible.
	\end{frame}
	
	\begin{frame}
		\frametitle{Minimax}
		Sea $d(i, j, k)$ el costo de la arista más pequeña entre las aristas más grandes de todos los caminos de $i$ hasta $j$ si como nodos intermedios solo se pueden usar los nodos $1 \ldots k$. \\
		\vfill
		$d(i, j, 0) =
		\left\{
			\begin{aligned}
				& 0       &\text{\quad si } i = j\\
				& C_{i,j} &\text{\quad si } (i, j) \in E\\
				& +\infty &\text{\quad en otro caso}
			\end{aligned}
		\right\}
		\text{ para } 
			\begin{aligned}
				& 1 \leq i \leq n\\
				& 1 \leq j \leq n\\
			\end{aligned}
		$
		\vfill
		$d(i, j, k) = \text{ min }
		\left\{
			\begin{aligned}
				& d(i, j, k-1)\\
				& \text{max }\left\{ d(i, k, k-1) + d(k, j, k-1) \right\}\\
			\end{aligned}
		\right\}
		$
		\\ \begin{center} Para $ 1 \leq i, j, k \leq n$ \end{center}
	\end{frame}
	
	\begin{frame}[fragile]
		\frametitle{Implementación}
		\begin{lstlisting}
			for (int i = 1; i <= n; ++i) {
			   for (int j = 1; j <= n; ++j){
			      d[i][j] = INF;
			   }
			   d[i][i] = 0;
			}

			for (int edge = 0; edge < m; ++edge){
			   int u, v, c; cin >> u >> v >> c;
			   d[u][v] = c;
			}
			
			for (int k = 1; k <= n; ++k){
			   for (int i = 1; i <= n; ++i){
			      for (int j = 1; j <= n; ++j){
			         d[i][j] = min( d[i][j] , max(d[i][k], d[k][j]) );
			      }
			   }
			}
		\end{lstlisting}
	\end{frame}
	
	\begin{frame}
		\frametitle{Maximin}
		\textcolor{blue}{\large Entrada}\\
		Un grafo con pesos $G = (V, E)$ \\ \quad \\
		\textcolor{blue}{\large Objetivo}\\
		Para cada pareja de nodos $(i, j)$ , hallar el camino de $i$ hasta $j$ donde la arista más pequeña del camino sea lo más grande posible.\\ \quad \\
		\textcolor{blue}{\large Ejemplos}\\
		Que el trayecto menos seguro sea lo más seguro posible.\\
		Que la autopista de menos carriles tenga la mayor cantidad de carriles.
	\end{frame}
	
	
	\begin{frame}[fragile]
		\frametitle{Implementación}
		\begin{lstlisting}
			for (int i = 1; i <= n; ++i) {
			   for (int j = 1; j <= n; ++j){
			      d[i][j] = -INF;
			   }
			   d[i][i] = INF;
			}

			for (int edge = 0; edge < m; ++edge){
			   int u, v, c; cin >> u >> v >> c;
			   d[u][v] = c;
			}
			
			for (int k = 1; k <= n; ++k){
			   for (int i = 1; i <= n; ++i){
			      for (int j = 1; j <= n; ++j){
			         d[i][j] = max( d[i][j] , min(d[i][k], d[k][j]) );
			      }
			   }
			}
		\end{lstlisting}
	\end{frame}
	
	\section{Tarea}
		\begin{frame}[fragile]
			\frametitle{Tarea}
			\begin{alertblock}{Tarea}
				Resolver los problemas de \url{http://contests.factorcomun.org/contests/55}
			\end{alertblock}
		\end{frame}

		\begin{frame}[fragile]
			\frametitle{Ayudas}
			\begin{block}{Problema A}
			
			\end{block}
			\begin{block}{Problema B}
				
			\end{block}
			\begin{block}{Problema C}
				
			\end{block}
		\end{frame}

\end{document}