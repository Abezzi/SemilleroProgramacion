\documentclass{beamer}
\usepackage[utf8]{inputenc}
\usepackage[spanish]{babel}
\usepackage{hyperref}
\usepackage{verbatim}
\usepackage{listings}
\usepackage{tikz}
\usepackage{animate} 
\usetikzlibrary{arrows}

\setbeamercovered{invisible}
\usetheme{Frankfurt}
\usefonttheme{serif}

% \setbeamertemplate{mini frames}{} % Para que no aparezcan las circulitos de las subsecciones

% Configurar los listings (Códigos)
\renewcommand{\lstlistingname}{Código}
\lstset{
	language=C++,               % Lenguaje
	basicstyle=\ttfamily\footnotesize,  % Tipo de fuente
	keywordstyle=\color{blue},  % Color de palabras clave
	stringstyle=\color{red},    % Color de strings
	commentstyle=\color{gray},  % Color de comentarios
	showstringspaces=false,     % No muestrar el _ cuando el string tiene espacios
	breaklines = true,          % Partir las líneas largas
	breakatwhitespace=true,	    % Partir las líneas en un espacio
	numbers=left,				% Numerar las líneas a la izq
	numberstyle=\tiny,			% Poner los números de las líneas pequeños
	numberblanklines=true,      % Numerar las líneas en blanco
	columns=fullflexible,       % No perder el formato al dejar los espacios
	keepspaces=true,   			% Dejar los espacios insertados
	frame=tb,					% Poner el recuadro
}

\AtBeginSection[]{%
  \begin{frame}<beamer>
    \frametitle{Contenido}
    \tableofcontents[sectionstyle=show/hide,subsectionstyle=hide/show/hide]
  \end{frame}
  \addtocounter{framenumber}{-1}% If you don't want them to affect the slide number
}

\title{Semillero de Programación}
\subtitle{Algoritmos de Teoría de Números}
\author{Ana Echavarría \and Juan Francisco Cardona}

\institute{Universidad EAFIT}
\date{17 de mayo de 2013}

\begin{document}

\begin{frame}
	\titlepage
\end{frame}

\begin{frame}
	\frametitle{Contenido}
	\tableofcontents
\end{frame}

\section[Problemas anteriores]{Problemas semana anterior}
	\begin{frame}
		\frametitle{Problema A - Internet Bandwidth}
		Hay que hallar el máximo flujo que puede enviarse desde una fuente a un sumidero
	\end{frame}
	
	\begin{frame}[fragile, allowframebreaks]
		\frametitle{Implementación}
		\begin{lstlisting}
			const int MAXN = 105;
			// Lista de adyacencia de la red residual. Solo las conexiones no 
			// los pesos. Se usa para que el BFS sea rapido
			vector <int> g [MAXN]; 
			int c [MAXN][MAXN]; // Capacidad de aristas de la red de flujos
			int f [MAXN][MAXN]; // El flujo de cada arista
			//El predecesor de cada nodo en el camino de aumentacion de s a t
			int prev [MAXN];

			void connect (int i, int j, int cap){
			    // Agregar SIEMPRE las dos aristas al grafo la red de flujos 
			    // sea dirigida. Esto es porque g representa la red residual
			    // que tiene aristas en los dos sentidos
			    g[i].push_back(j);
			    g[j].push_back(i);
			    c[i][j] += cap;    
			    // Omitir esta linea si el grafo es dirigido
			    c[j][i] += cap; 
			}

			int maxflow(int s, int t, int n){
			    // Inicializar el flujo en 0
			    for (int i = 0; i <= n; i++)
			        for (int j = 0; j <= n; j++)
			            f[i][j] = 0;

			    int flow = 0;
			    while (true){
			        // Encontrar camino entre s y t con bfs
			        for (int i = 0; i <= n; i++) prev[i] = -1;

			        queue <int> q;
			        q.push(s);
			        prev[s] = -2;  // s no tiene nodo anterior en el camino




			        while (q.size() > 0){
			            int u = q.front(); q.pop();
			            if (u == t) break;
			            for (int i = 0; i < g[u].size(); ++i){
			                int v = g[u][i];
			                // Si el nodo no esta visitado
			                // y el peso en la red residual es > 0
			                if (prev[v] == -1 and c[u][v] - f[u][v] > 0){
			                    q.push(v);
			                    prev[v] = u;
			                }
			            }
			        }
			        // Si no se llego a t en el camino es porque no hay mas 
			        // caminos de aumentacion y el ciclo se detiene
			        if (prev[t] == -1) break;






			        // Hallar el cuello de botella
			        // Se recorre el camino desde t hasta s
			        int extra = 1 << 30;
			        int end = t;
			        while (end != s){
			            int start = prev[end];
			            // El cuello de botella es el minimo peso del camino 
			            // en la red residual. El peso en la red residual es 
			            // la capacidad de la arista menos el flujo
			            extra = min(extra, c[start][end] - f[start][end]);
			            end = start;
			        }






			        // Bombear el flujo extra por la arista
			        end = t;
			        while (end != s){
			            int start = prev[end];
			            f[start][end] += extra;
			            f[end][start] = -f[start][end];
			            end = start;
			        }

			        // Agregar el flujo enviado a la respuesta
			        flow += extra;
			    }

			    return flow;
			}

			int main(){
			    int run = 1;
			    int n;
			    while (cin >> n){
			        if (n == 0) break;

			        // Limpiar el grafo y la capacidad
			        for (int i = 0; i <= n; i++){
			            g[i].clear();
			            for (int j = 0; j <= n; j++){
			               c[i][j] = 0; 
			            }
			        }

			        // Leer el grafo
			        int s, t, edges;
			        cin >> s >> t >> edges;
			        for (int i = 0; i < edges; i++){
			            int n1, n2, cap;
			            cin >> n1 >> n2 >> cap;
			            connect(n1, n2, cap);
			        }

			        // Imprimir la respuesta
			        printf("Network %d\n", run++);
			        printf("The bandwidth is %d.\n\n", maxflow(s, t, n));

			    }
			    return 0;
			}
		\end{lstlisting}
	\end{frame}
	
	\begin{frame}
		\frametitle{Problema B - Angry Programmer}
		\begin{itemize}
			\item Usar el algoritmo de máximo flujo para hallar el mínimo corte
			\item Dividir cada nodo en 2 para poder tener en cuenta la capacidad de los nodos.
			\item Para que no se puedan destruir la fuente y el sumidero ponerles capacidad infinita.
		\end{itemize}
	\end{frame}
	
	\begin{frame}[fragile, allowframebreaks]
		\frametitle{Implementación}
		\begin{lstlisting}
			const int MAXN = 2 * 50 + 5;
			const int INF = INT_MAX;
			vector < int > g [MAXN];
			int c [MAXN][MAXN];
			int f [MAXN][MAXN];
			int prev [MAXN];

			void refresh (int n){
			    for (int i = 0; i < 2 * n; i++){
			        g[i].clear();
			        for (int j = 0; j < 2 * n; j++){
			            f[i][j] = 0;
			            c[i][j] = 0;
			        }
			    }
			}


			void join (int u, int v, int cost){
			    int in_u = 2 * u;
			    int out_u = 2 * u + 1;

			    int in_v = 2 * v;
			    int out_v = 2 * v + 1;

			    g[out_u].push_back(in_v);
			    g[in_v].push_back(out_u);
			    c[out_u][in_v] += cost;

			    g[out_v].push_back(in_u);
			    g[in_u].push_back(out_v);
			    c[out_v][in_u] += cost;
			}




			void join_self (int u, int cost){
			    g[2 * u].push_back(2 * u + 1);
			    g[2 * u + 1].push_back(2 * u);
			    c[2 * u][2 * u + 1] += cost;
			    c[2 * u + 1][2 * u] += cost;
			}

			bool path(int s, int t){
			    for (int i = 0; i <= t; i++)
			        prev[i] = -2;

			    queue <int> q;
			    q.push(s);
			    prev[s] = -1;
			    while (q.size() > 0){
			        int u = q.front();
			        q.pop();
			        if (u == t) return true;
			        for (int i = 0; i < g[u].size(); i++){
			            int v = g[u][i];
			            if (prev[v] == -2 and c[u][v] - f[u][v] > 0){
			                q.push(v);
			                prev[v] = u;
			            }
			        }
			    }
			    return false;
			}

			int bottleneck(int s, int t){
			    int curr = t;
			    int ans = INF;
			    while (curr != s){
			        ans = min (ans, c[prev[curr]][curr] - f[prev[curr]][curr]);
			        curr = prev[curr];
			    }
			    assert(prev[s] == -1);
			    return ans;
			}

			void pump (int s, int t, int extra){
			    int curr = t;
			    while (curr != s){
			        f[prev[curr]][curr] += extra;
			        f[curr][prev[curr]] -= extra;
			        curr = prev[curr];
			    }
			}

			int max_flow(int s, int t){
			    int ans = 0;
			    while (true){
			        // Find path
			        if (!path(s, t)) break;
			        // Find bottleneck
			        int extra = bottleneck(s, t);
			        // Pump bottleneck
			        pump(s, t, extra);
			        // Add flow to answer
			        ans += extra;
			    }
			    return ans;
			}

			int main(){
			    int nodes, edges;
			    while (scanf("%d %d", &nodes, &edges) == 2){
			        if (nodes == 0 and edges == 0) break;

			        refresh(nodes);
			        // Crear los nodos (entrada y salida)
			        for (int i = 1; i < nodes - 1; i++){
			            int j, cost; 
			            cin >> j >> cost;
			            j--;
			            join_self(j, cost);
			        }
			        // El peso de la fuente y el sumidero es INF
			        join_self(0, INF);
			        join_self(nodes - 1, INF);

			        // Unir las aristas
			        for (int i = 0; i < edges; i++){
			            int u, v, cost;
			            cin >> u >> v >> cost;
			            u--; v--;
			            join(u, v, cost);
			        }

			        cout << max_flow(0, 2 * nodes - 1) << endl;
			    }
			    return 0;
			}
		\end{lstlisting}
	\end{frame}

\section[Divisores]{Hallar los divisores de un número}
	
	\begin{frame}
		\frametitle{Divisores de un número}
		\begin{itemize}
			\item Los divisores vienen de a parejas, si $a | b$ entonces $\frac{b}{a} | b$
			\item Es por esto que basta con mirar hasta la raíz cuadrada del número para hallar sus divisores
		\end{itemize}
	\end{frame}
	
	\begin{frame}[fragile]
		\frametitle{Divisores de un número}
		\begin{lstlisting}
			void divisors(int n){
			    int i;
			    for (i = 1; i * i < n; ++i){
			        if (n % i == 0) printf("%d\n%d\n", i, n/i);
			    }
			    // Si el numero es un cuadrado perfecto, imprimir su raiz cuadrara una sola vez
			    if (i * i == n) printf("%d\n", i);
			}
		\end{lstlisting}
	\end{frame}
	
	\begin{frame}
		\frametitle{Complejidad}
		\begin{alertblock}{Pregunta}
			¿Cuántos números hay que revisar para hallar todos los divisores de un número $n$?\\
			De acuerdo a eso, ¿cuál es la complejidad del algoritmo?
		\end{alertblock}
		\pause
		\begin{exampleblock}{Respuesta}
			La complejidad de este algoritmo es $O(\sqrt{n})$
		\end{exampleblock}
	\end{frame}

\section[GCD y LCM]{Máximo común divisor y mínimo común múltiplo}

	\begin{frame}[fragile]
		\frametitle{Máximo común divisor}
		\begin{block}{GCD}
			Para hallar el máximo común divisor entre dos números \verb|a| y \verb|b| ejecutar el comando \verb|__gcd(a, b)|
		\end{block}
	\end{frame}
	
	\begin{frame}
		\frametitle{Mínimo común múltiplo}
		\begin{block}{LCM}
			El mínimo común múltiplo se puede hallar en términos del máximo común divisor así:
			$$ \operatorname{lcm}(a, b) = \frac{|a \cdot b|}{\operatorname{gcd}(a,b)}$$
		\end{block}
	\end{frame}
	
	\begin{frame}
		\frametitle{Propiedades}
		\begin{itemize}
			\item Si $c|a$ y $c|b$ entonces $c|\operatorname{gcd}(a, b)$
			\item Si $a|b\cdot c$ y $\operatorname{gcd}(a,b) = 1$ entonces $a|c$
			\item $\operatorname{gcd}(a,b)$ es una combinación lineal entera de $a$ y $b$
		\end{itemize}
	\end{frame}

\section[Criba]{Criba de Eratóstenes}

	\begin{frame}
		\frametitle{Criba de Eratóstenes}
		\begin{itemize}
			\item Es un algoritmo para hallar todos los números primos hasta un límite $n$
			\item Inicialmente asume marca todos los números de 2 hasta $n$ como no visitados y el 1 como visitado
			\item Empieza a recorrer los números uno por uno
			\item Cuando encuentra un número no visitado empieza a marcar todos sus múltiplos (no él) como visitados
			\item Cuando llega al número $n$, los números que estén marcados como no visitados son los números primos del intervalo $[1, n]$
		\end{itemize}
	\end{frame}
	
	\begin{frame}
		\frametitle{Criba de Eratóstenes}
		\begin{center}
			\includegraphics<1>[height = 0.8\textheight]{./gif/Sieve_1.png}
			\includegraphics<2>[height = 0.8\textheight]{./gif/Sieve_2.png}
			\includegraphics<3>[height = 0.8\textheight]{./gif/Sieve_3.png}
			\includegraphics<4>[height = 0.8\textheight]{./gif/Sieve_4.png}
			\includegraphics<5>[height = 0.8\textheight]{./gif/Sieve_5.png}
			\includegraphics<6>[height = 0.8\textheight]{./gif/Sieve_6.png}
			\includegraphics<7>[height = 0.8\textheight]{./gif/Sieve_7.png}
			\includegraphics<8>[height = 0.8\textheight]{./gif/Sieve_8.png}
			\includegraphics<9>[height = 0.8\textheight]{./gif/Sieve_9.png}
			\includegraphics<10>[height = 0.8\textheight]{./gif/Sieve_10.png}
			\includegraphics<11>[height = 0.8\textheight]{./gif/Sieve_11.png}
		\end{center}
	\end{frame}
	
	\begin{frame}[fragile]
		\begin{lstlisting}
			const int MAXN = 1000000;
			bool sieve[MAXN + 5];
			vector <int> primes;

			void build_sieve(){
			    memset(sieve, false, sizeof(sieve));
			    sieve[0] = sieve[1] = true;

			    for (int i = 2; i * i <= MAXN; i ++){
			        if (!sieve[i]){
			            for (int j = i * i; j <= MAXN; j += i){
			                sieve[j] = true;
			            }
			        }
			    }
			    for (int i = 2; i <= MAXN; ++i){
			        if (!sieve[i]) primes.push_back(i);
			    }
			}
		\end{lstlisting}
	\end{frame}
	
	\begin{frame}
		\frametitle{Complejidad}
		\begin{alertblock}{Pregunta}
			¿Cuántas veces se visita cada número?\\
			De acuerdo a eso, ¿cuál es la complejidad de la criba?
		\end{alertblock}
		\pause
		\begin{exampleblock}{Respuesta}
			Como cada número se visita una única vez, la complejidad del algoritmo es $O(n)$.
		\end{exampleblock}
	\end{frame}

\section[Factorización prima]{Hallar la factorización prima de un número}

	\begin{frame}
		\frametitle{Factorización prima}
		\begin{alertblock}{Pregunta}
			Si se conocen todos los primos desde 0 hasta $\sqrt{a}$ ¿cómo usarlos para hallar la factorización prima de $a$?
		\end{alertblock}
		\pause
		\begin{exampleblock}{Respuesta}
			\begin{itemize}
				\item La factorización prima de $a$ contiene máximo un solo primo mayor que $\sqrt{a}$
				\item Se divide $a$ por todos los primos menores o iguales a su raíz cuadrada
				\item Hay que tener en cuenta que un primo puede aparecer varias veces en la factorización
				\item Si luego de dividirlo queda un número mayor que 1, ese número es primo
			\end{itemize}
		\end{exampleblock}
	\end{frame}

	\begin{frame}[fragile]
		\frametitle{Factorización prima}
		\begin{lstlisting}
			const int MAXN = 1000000; // MAXN > sqrt(a)
			bool sieve[MAXN + 5];
			vector <int> primes;
			
			vector <long long> factorization(long long a){
			    // Se asume que se tiene y se llamo la funcion build_sieve()
			    vector <long long> ans;
			    long long b = a;
			    for (int i = 0; 1LL * primes[i] * primes[i] <= a; ++i){
			        int p = primes[i];
			        while (b % p == 0){
			            ans.push_back(p);
			            b /= p;
			        }
			    }
			    if (b != 1) ans.push_back(b);
			    return ans;
			}
		\end{lstlisting}
	\end{frame}
	
	\begin{frame}
		\frametitle{Complejidad}
		\begin{block}{Complejidad}
			La complejidad del algoritmo para hallar la factorización prima de un número es aproximadamente $O(\sqrt{n})$
		\end{block}
	\end{frame}


\section[Bigmod]{Exponenciación logarítmica}

	\begin{frame}
		\frametitle{La operación módulo}
		La operación $a \operatorname{mod} n$ es el residuo que queda de dividir $a$ por $n$.
		\begin{block}{Propiedades}
			\begin{itemize}
				\item $(a \operatorname{mod} n) \operatorname{mod} n = a \operatorname{mod} n$
				\item $(a + b) \operatorname{mod} n = ((a \operatorname{mod} n) + (b\operatorname{mod} n)) \operatorname{mod} n$
				\item $(a \cdot b) \operatorname{mod} n = ((a \operatorname{mod} n) \cdot (b\operatorname{mod} n)) \operatorname{mod} n$
				\item $\displaystyle\left(\frac{a}{b} \right) \operatorname{mod} n \neq \left(\frac{a \operatorname{mod} n}{b\operatorname{mod} n}\right) \operatorname{mod} n$
			\end{itemize}
		\end{block}
	\end{frame}
	
	\begin{frame}
		\frametitle{La operación módulo}
		\begin{alertblock}{Pregunta}
			¿Cómo utilizar el módulo para hallar los últimos $k$ dígitos del número $a$?
		\end{alertblock}
		\pause
		\begin{exampleblock}{Respuesta}
			Los últimos $k$ dígitos del número $a$ se pueden hallar con la operación $a \operatorname{mod} 10^k$
		\end{exampleblock}
	\end{frame}

	\begin{frame}[fragile]
		\frametitle{Exponenciación logarítmica}
		\begin{alertblock}{Bigmod}
			Se quiere hallar el valor de $b^{\,p} \operatorname{mod} m$ para valores de $b$ y de $p$ muy grandes ($b^{\,p}$ no cabe en un \verb|unsigned long long|).
			¿Cómo hacer eso eficientemente?
		\end{alertblock}
		\pause
		\begin{exampleblock}{Algoritmo}
			El valor de $b^{\,p} \operatorname{mod} m = \operatorname{bigmod}(b, p, m)$ se puede calcular en orden $\operatorname{log}(p)$ de la siguiente manera:
			$$ \operatorname{bigmod}(b, p, m) =
			\begin{cases}
				1    & \text{si } p = 0\\
				\big(\operatorname{bigmod}(b, \frac{p}{2}, m)\big)^2 \operatorname{mod} \,m &  \text{si } p \text{ es par}\\
				\big(b \cdot \operatorname{bigmod}(b, p-1, m)\big)   \operatorname{mod} \,m &  \text{si } p \text{ es impar}
			\end{cases}$$
		\end{exampleblock}
	\end{frame}

	\begin{frame}[fragile]
		\frametitle{Bigmod}
		\begin{lstlisting}
			int bigmod(int b, int p, int m){
			    if (p == 0) return 1;
			    if (p % 2 == 0){
			        int mid = bigmod(b, p/2, m);
			        return (1LL * mid * mid) % m;
			    }else{
			        int mid = bigmod(b, p-1, m);
			        return (1LL * mid * b) % m;
			    }
			}
		\end{lstlisting}
	\end{frame}

\section[Combinatoria]{Combinatoria}

	\begin{frame}
		\frametitle{Coeficientes binomiales}
		El valor de $\displaystyle\binom{n}{k}$ que es el número de subconjuntos de $k$ elementos escogidos de un conjunto con $n$ elementos se puede definir como:
		$$ \binom{n}{k} = 
		\begin{cases}
			\,\, 1 & \text{si } n = k \\
			\,\, 1 & \text{si } k = 0 \\
			\displaystyle\binom{n-1}{k-1} + \binom{n-1}{k} & \text{si } k < n
		\end{cases}
		$$
		Nótese que $\displaystyle\binom{n}{k}$ está definido solo si $k \leq n$.
	\end{frame}
	
	\begin{frame}[fragile]
		\frametitle{Coeficientes binomiales}
		\begin{lstlisting}
			const int MAXN = 30;
			long long choose[MAXN][MAXN];

			void build_binomial(int N){
			    for (int n = 0; n <= N; ++n) choose[n][0] = choose[n][n] = 1;

			    for (int n = 1; n <= N; ++n){
			        for (int k = 1; k < n; ++k){
			            choose[n][k] = choose[n-1][k-1] + choose[n-1][k];
			        }
			    }
			}
		\end{lstlisting}
	\end{frame}
	
	\begin{frame}
		\frametitle{Complejidad}
		\begin{alertblock}{Pregunta}
			¿Cuál es la complejidad del algoritmo para hallar los coeficientes binomiales?
		\end{alertblock}
		\pause
		\begin{exampleblock}{Respuesta}
			El algoritmo tiene una complejidad de $O(n^2)$
		\end{exampleblock}
	\end{frame}
	
	\begin{frame}
		\frametitle{Permutaciones}
		Una permutación de un conjunto es un ordenamiento particular de sus elementos.
		\quad \\ \quad \\
		\begin{block}{Permutaciones de elementos diferentes}
			El número de permutaciones de $n$ elementos diferentes es $n!$
		\end{block}
	\end{frame}
	
	\begin{frame}
		\frametitle{Permutaciones}
		
		\begin{block}{Permutaciones con elementos repetidos}
			El número de permutaciones de $n$ elementos donde hay $m_1$ elementos repetidos de tipo 1, $m_2$ elementos repetidos de tipo 2,~\ldots, $m_k$ elementos repetidos de tipo $k$ es
			$$ \frac{n!}{m_1! m_2! \cdots m_k!} $$
		\end{block}
		
		\begin{exampleblock}{Ejemplo}
			¿De cuántas formas diferentes se pueden reordenar los números 313?\\
			\pause
			Se pueden reordenar de $\displaystyle\frac{3!}{1!\,2!} = 3 $ maneras diferentes
		\end{exampleblock}
	\end{frame}

	\begin{frame}
		\frametitle{Permutaciones}
		\begin{block}{Permutaciones de subconjuntos}
			El número de permutaciones de $k$ elementos diferentes tomados de un conjunto de $n$ elementos es
			$$ \frac{n!}{(n-k)!} = k! \binom{n}{k}$$
		\end{block}
	\end{frame}


\section{Tarea}
	\begin{frame}[fragile]
		\frametitle{Tarea}
		\begin{alertblock}{Tarea}
			Resolver los problemas de \url{http://contests.factorcomun.org/contests/60}
		\end{alertblock}
	\end{frame}

\end{document}