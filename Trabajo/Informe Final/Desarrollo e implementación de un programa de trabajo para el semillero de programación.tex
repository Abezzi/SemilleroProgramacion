
\documentclass[conference]{IEEEtran}


\usepackage{cite}
% cite.sty was written by Donald Arseneau
% V1.6 and later of IEEEtran pre-defines the format of the cite.sty package
% \cite{} output to follow that of IEEE. Loading the cite package will
% result in citation numbers being automatically sorted and properly
% "compressed/ranged". e.g., [1], [9], [2], [7], [5], [6] without using
% cite.sty will become [1], [2], [5]--[7], [9] using cite.sty. cite.sty's
% \cite will automatically add leading space, if needed. Use cite.sty's
% noadjust option (cite.sty V3.8 and later) if you want to turn this off.
% cite.sty is already installed on most LaTeX systems. Be sure and use
% version 4.0 (2003-05-27) and later if using hyperref.sty. cite.sty does
% not currently provide for hyperlinked citations.
% The latest version can be obtained at:
% http://www.ctan.org/tex-archive/macros/latex/contrib/cite/
% The documentation is contained in the cite.sty file itself.


\usepackage[cmex10]{amsmath}
\interdisplaylinepenalty=2500

\usepackage[tight,footnotesize]{subfigure}

\usepackage{url}
% Read the url.sty source comments for usage information. Basically,
% \url{my_url_here}.
\usepackage[spanish]{babel}
\renewcommand{\IEEEkeywordsname}{Palabras claves}
\usepackage[utf8]{inputenc}
\usepackage{hyperref}

\hyphenation{}


\begin{document}
%
% paper title
% can use linebreaks \\ within to get better formatting as desired
\title{Desarrollo e implementación de un programa de trabajo para el Semillero de Programación}


% author names and affiliations
% use a multiple column layout for up to three different
% affiliations
\author{\IEEEauthorblockN{A. Echavarría Uribe}
\IEEEauthorblockA{Ingeniería Matemática\\
Universidad EAFIT\\
Medellín, Colombia\\
Email: aechava3@eafit.edu.co}
\and
\IEEEauthorblockN{J. F. Cardona Mc'Cormick}
\IEEEauthorblockA{Escuela de Ingeniería\\
Universidad EAFIT\\
Medellín, Colombia\\
Email: fcardona@eafit.edu.co}}


\maketitle


\begin{abstract}
% Se presenta en un párrafo de máximo 200 palabras, en el que se señale el motivo de la realización del proyecto, el contexto dentro del cual se desarrolló y la contribución de acuerdo con los resultados obtenidos sin detallar en los mismos. El resumen debe presentarse en español y en inglés.
\end{abstract}

\begin{keywords}
% Se citan las palabras que pueden servir para identificar el trabajo en las bases de datos en caso de publicación y que hacen referencia a los temas de los que trata el trabajo.
\end{keywords}

\section{Introducción}
% En ella se sintetizan los antecedentes, justificación y alcance del trabajo de forma que permita al lector crear una imagen mental del enfoque del proyecto y los resultados que encontrará más adelante en el texto.
El Semillero de Programación es un grupo de la Universidad EAFIT en el que los estudiantes con interés en la programación, las matemáticas y los algoritmos tienen un espacio para aprender acerca de estos temas y prepararse para participar en las maratones de programación realizadas por ACIS/REDIS \cite{ACIS} y por la ACM-ICPC \cite{ICPC}. En este semillero se enseñan los temas más útiles \cite{ProgrammingChallenges, Halim, Halim2, Brasil} para estas competencias: los algoritmos de grafos, strings y teoría de números, programación dinámica, la recursividad y las estructuras de datos.\\
Durante los últimos años el Semillero ha estado a cargo de estudiantes destacados en las maratones de programación bajo la supervisión de docentes del Departamento de Ingeniería de Sistemas y, a pesar de que se han trabajado y discutido algoritmos, temas y problemas, no se desarrolló un plan de trabajo para este grupo. Dos consecuencias de lo anterior son que los miembros del Semillero a veces no tenían la fundamentación necesaria para aprender algunos de los algoritmos más complejos y que cuando algún estudiante nuevo se encargaba del Semillero no sabía cuáles eran los conocimientos que los estudiantes tenían y qué temas nuevos por discutir se ajustaban a su nivel. Por lo anterior se decidió crear, desarrollar e implementar un plan de trabajo para el Semillero de modo que este tuviera una estructura de trabajo que permita la apropiación progresiva del conocimiento necesario para poder resolver los problemas de las maratones de programación y que proporcione a quienes se encargarán del Semillero en versiones futuras información acerca de los conocimientos que tienen los estudiantes.\\
Actualmente, los estudiantes que pertenecen al Semillero de Programación son en su mayoría de tercer semestre, lo que quiere decir que tienen conocimientos acerca de cómo programar mas no conocen las técnicas más utilizadas en la solución de problemas de maratones de programación como los son los algoritmos de grafos, strings, la programación dinámica y los conceptos y algoritmos básicos de teoría de números. El programa desarrollado busca ajustarse a este nivel para poder proporcionar a los estudiantes la fundamentación teórica necesaria para resolver los problemas más comunes presentados en las maratones de programación. En el programa de trabajo se desarrolló documentación, diapositivas y competencias que servirán como material de trabajo para futuras generaciones del semillero.\\
Se espera que los temas enseñados a los estudiantes durante este semestre y el próximo sirvan para que ellos tengan un buen desempeño en la Maratón Nacional de Programación ACIS/REDIS que se realiza en octubre y tengan la posibilidad de participar en la Maratón Regional Suramericana ACM-ICPC y en la Maratón Mundial ACM-ICPC bajo el nombre la Universidad.


\section{Metodología}
% Este ítem debe presentar el proceso metodológico seguido durante el desarrollo de las diferentes etapas del proyecto.


\section{Resultados}
% Se reportan los resultados, sin emitir juicios ni opiniones de carácter personal. Se deben definir los símbolos y las unidades empleadas, incluir figuras y tablas que reporten los resultados obtenidos. Los datos se presentan en forma  tal que otras personas los puedan usar, haciendo énfasis en el texto de aquellos aspectos que son más importantes en las tablas, gráficos o figuras. Dar barras de errores o límites de validez para datos numéricos o gráficos. Las tablas y figuras deben ser poseer un título que describa de manera concisa el contenido de lo que se cita. Igualmente, en este ítem se extraen los principios, relaciones, o generalizaciones de los resultados logradas como consecuencia del análisis de comportamiento de los mismos.

\section{Conclusión}
% Condensa los avances o aportes que en conocimiento dan los resultados obtenidos luego del desarrollo del proyecto.

\section{Agradecimientos}
% En este ítem se citan las personas o entidades gracias a las cuales pudo desarrollarse el proyecto gracias a sus contribuciones (financieras o en especie). Se deben señalar los nombres completos y las afiliaciones. Ejemplo: “I wish to thank Prof. L. M. Brown of the Cavendish Laboratory, Cambridge, for suggestion this review, and to acknowledge my debt to the books listed below”.

\nocite{*}
\bibliographystyle{IEEEtran}
\bibliography{Bibliografia}
	% Se deben citar todas las fuentes de información de las cuales se hayan tomado datos útiles para la investigación. El formato más empleado presenta la referencia en el siguiente orden: Autores (apellido, inicial del primer nombre si son varios autores separados por coma igualmente); Título del artículo; Revista; Número; Volumen; páginas


\section*{Anexos}






\end{document}


