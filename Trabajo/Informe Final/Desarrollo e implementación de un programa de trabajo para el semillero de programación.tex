
\documentclass[conference]{IEEEtran}


\usepackage{cite}
% cite.sty was written by Donald Arseneau
% V1.6 and later of IEEEtran pre-defines the format of the cite.sty package
% \cite{} output to follow that of IEEE. Loading the cite package will
% result in citation numbers being automatically sorted and properly
% "compressed/ranged". e.g., [1], [9], [2], [7], [5], [6] without using
% cite.sty will become [1], [2], [5]--[7], [9] using cite.sty. cite.sty's
% \cite will automatically add leading space, if needed. Use cite.sty's
% noadjust option (cite.sty V3.8 and later) if you want to turn this off.
% cite.sty is already installed on most LaTeX systems. Be sure and use
% version 4.0 (2003-05-27) and later if using hyperref.sty. cite.sty does
% not currently provide for hyperlinked citations.
% The latest version can be obtained at:
% http://www.ctan.org/tex-archive/macros/latex/contrib/cite/
% The documentation is contained in the cite.sty file itself.


\usepackage[cmex10]{amsmath}
\interdisplaylinepenalty=2500

\usepackage[tight,footnotesize]{subfigure}

\usepackage{url}
% Read the url.sty source comments for usage information. Basically,
% \url{my_url_here}.
\usepackage[spanish]{babel}
\renewcommand{\IEEEkeywordsname}{Palabras claves}
\addto\captionsspanish{\renewcommand{\tablename}{Tabla}}
\usepackage[utf8]{inputenc}
\usepackage{hyperref}
\usepackage{color}


\hyphenation{}


\begin{document}
%
% paper title
% can use linebreaks \\ within to get better formatting as desired
\title{Desarrollo e implementación de un programa de trabajo para el Semillero de Programación}


% author names and affiliations
% use a multiple column layout for up to three different
% affiliations
\author{\IEEEauthorblockN{A. Echavarría Uribe}
\IEEEauthorblockA{Ingeniería Matemática\\
Universidad EAFIT\\
Medellín, Colombia\\
Email: aechava3@eafit.edu.co}
\and
\IEEEauthorblockN{J. F. Cardona Mc'Cormick}
\IEEEauthorblockA{Escuela de Ingeniería\\
Universidad EAFIT\\
Medellín, Colombia\\
Email: fcardona@eafit.edu.co}}


\maketitle


\begin{abstract}
% Se presenta en un párrafo de máximo 200 palabras, en el que se señale el motivo de la realización del proyecto, el contexto dentro del cual se desarrolló y la contribución de acuerdo con los resultados obtenidos sin detallar en los mismos. El resumen debe presentarse en español y en inglés.
El Semillero de Programación de la Universidad EAFIT busca preparar a los estudiantes para las maratones de programación organizadas por ACIS/REDIS y por la ACM-ICPC. En este trabajo se muestra cómo se desarrolló y se trabajó un programa de clases para este Semillero buscando preparar a estudiantes novatos para las maratones de programación. 
\end{abstract}

\begin{keywords}
% Se citan las palabras que pueden servir para identificar el trabajo en las bases de datos en caso de publicación y que hacen referencia a los temas de los que trata el trabajo.
Semillero de Programación, Maratones de Programación, ACM-ICPC, ACIS/REDIS, Curso de programación competitiva.
\end{keywords}

\section{Introducción}
% En ella se sintetizan los antecedentes, justificación y alcance del trabajo de forma que permita al lector crear una imagen mental del enfoque del proyecto y los resultados que encontrará más adelante en el texto.
El Semillero de Programación es un grupo de la Universidad EAFIT en el que los estudiantes con interés en la programación, las matemáticas y los algoritmos tienen un espacio para aprender acerca de estos temas y prepararse para participar en las maratones de programación realizadas a nivel nacional por ACIS/REDIS \cite{ACIS} y a nivel internacional por la ACM-ICPC \cite{ICPC}. En este Semillero se enseñan los temas más útiles \cite{Halim, Halim2, ProgrammingChallenges, Brasil} para estas competencias: los algoritmos de grafos, strings y teoría de números, programación dinámica, la recursividad y las estructuras de datos.\\
Durante los últimos años el Semillero ha estado a cargo de estudiantes destacados en las maratones de programación bajo la supervisión de docentes del Departamento de Ingeniería de Sistemas y, a pesar de que se han trabajado y discutido algoritmos, temas y problemas, no se desarrolló un plan de trabajo para este grupo. Dos consecuencias de lo anterior son que los miembros del Semillero a veces no tenían la fundamentación necesaria para aprender algunos de los algoritmos más complejos y que cuando algún estudiante nuevo se encargaba del Semillero no sabía cuáles eran los conocimientos que los estudiantes tenían y qué temas nuevos por discutir se ajustaban a su nivel. Por lo anterior se decidió crear, desarrollar e implementar un plan de trabajo para el Semillero de modo que este tenga una estructura que permita la apropiación progresiva del conocimiento necesario para poder resolver los problemas de las maratones de programación y que proporcione a quienes se encargarán del Semillero en versiones futuras información acerca de los conocimientos que tienen los estudiantes.\\
Actualmente, los estudiantes que pertenecen al Semillero de Programación son en su mayoría de tercer semestre, lo que quiere decir que tienen conocimientos acerca de cómo programar mas no conocen las técnicas más utilizadas en la solución de problemas de maratones de programación como los son los algoritmos de grafos, strings, la programación dinámica y los conceptos y algoritmos básicos de teoría de números.\\ %El programa desarrollado busca ajustarse a este nivel para poder proporcionar a los estudiantes la fundamentación teórica necesaria para resolver los problemas más comunes presentados en las maratones de programación. En el programa de trabajo se desarrolló documentación, diapositivas y competencias que servirán como material de trabajo para futuras generaciones del Semillero.\\
Se espera que los temas enseñados a los estudiantes durante este semestre y el próximo sirvan para que ellos tengan un buen desempeño en la Maratón Nacional de Programación ACIS/REDIS que se realiza en octubre y tengan la posibilidad de participar en la Maratón Regional Suramericana ACM-ICPC y en la Maratón Mundial ACM-ICPC bajo el nombre la Universidad com ha ocurrido en ocasiones anteriores.

\section{Metodología}
% Este ítem debe presentar el proceso metodológico seguido durante el desarrollo de las diferentes etapas del proyecto.
El plan de trabajo del Semillero está basado principalmente en la metodología utilizada por Steven Halim en el curso Competitive Programming de la Universidad Nacional de Singapur (NUS). Este curso, al igual que el Semillero, busca preparar a los estudiantes para las competencias de la ACM-ICPC y está incorporado como curso electivo para estudiantes de matemáticas, ciencias de la computación e ingeniería electrónica de tercer año con conocimientos previos de programación, algoritmos y estructuras de datos. En el curso se trabajan algoritmos de nivel medio y avanzado pero, a diferencia de un curso de algoritmos corriente, se hace énfasis en cómo implementarlos eficientemente y en sus aplicaciones a las competencias de programación, en lugar de enfocarse en las demostraciones de su corrección y el análisis formal de su complejidad\cite{PaperHalim}.\\
El método de enseñanza del curso de la NUS consiste tener clases teóricas en las cuales se enseñen los algoritmos a trabajar y hacer las evaluaciones por medio de competencias en las que los problemas, tomados de archivos históricos de competencias anteriores de la ICPC, tengan relación con los temas enseñados. Las dos razones principales por la que decidieron trabajar con competencias en lugar de exámenes son preparar a los estudiantes para las competencias oficiales de la ACM-ICPC y motivarlos a ser cada vez mejores al ponerlos a medir sus habilidades contra los demás estudiantes. Este método ha resultado beneficioso para los estudiantes de la NUS y el Profesor Halim cree que es porque incita a los estudiantes a estudiar más para mejorar ya que buscar ser los mejores es es una característica natural de los humanos.\\
La principal ventaja que ha tenido este curso en la NUS ha sido darle a los estudiantes talentosos e interesados en la programación un espacio en el que puedan prepararse para las competencias y conocer otras personas interesadas en este tema con las que puedan discutir, competir y formar equipos que tengan buen desempeño en las competencias de la ACM-ICPC, llegando a participar en la Maratón Mundial ACM-ICPC \cite{AlgoNUS}. El Semillero de Programación de EAFIT tiene el mismo objetivo y es por esto que se decidió desarrollar un plan de trabajo basado en esta metodología y adaptarlo para el nivel del Semillero que va dirigido a estudiantes de tercer semestre y no de tercer año como lo es en la NUS.
	\subsection{Contenido y estructura de las sesiones}
	Para el desarrollo del plan de trabajo fue necesario escoger el contenido a trabajar en cada sesión de manera que se trataran de cubrir la mayoría de las técnicas básicas e intermedias de programación requeridas para las maratones. La elección de los temas si hizo basándose en el nivel de los estudiantes del Semillero, el material de los cursos ``Competitive Programming'' de la NUS \cite{CourseNUS} y ``Escuela de verano para maratones de programación'' de la Universidad Estatal de Campinas \cite{Brasil}, los temas de varios libros acerca de algoritmos y de competencias de programación \cite{Cormen, ProgrammingChallenges, Halim, Halim2, ArtOfProgramming} y otros temas que los autores consideraron importantes de acuerdo a su experiencia obtenida gracias a la participación en diferentes competencias de programación.\\
	Cada sesión de Semillero, independiente del tema que se trabaje, consiste en tres partes principales: 
	\begin{enumerate}
		\item Discusión y solución de los problemas propuestos como tarea en la sesión anterior.\\Esto se hace con el fin de que los estudiantes, luego de haber intentado resolver los problemas propuestos de manera individual, entiendan su solución y de esta manera aprendan de ella. Para los estudiantes que lograron resolver los problemas este es un espacio en el que pueden compartir su solución con sus compañeros y ver una implementación diferente del problema que resolvieron; esto último les permite conocer diferentes formas de desarrollar y pensar en un mismo algoritmo y posiblemente conocer funciones y métodos del lenguaje C++ que hacen las implementaciones más cortas y sencillas.
		\item Exposición del nuevo tema a trabajar, mostrando los algoritmos, los elementos matemáticos relacionados y sus implementaciones en el lenguaje C++ \cite{C++}.\\La forma de abordar los temas para llevar al estudiante al entendimiento del algoritmo está basada principalmente en las métodos de enseñanza manejados en los cursos Algorithms: Design and Analysis Part 1/2 \cite{Coursera1, Coursera2} e Introduction to Algorithms \cite{CourseMIT} y Competitive Programming \cite{CourseNUS}. Por otra parte, la implementación de los algoritmos se basó en las implementaciones mostradas en \cite{Halim, Halim2, ArtOfProgramming} con el fin de tener implementaciones eficientes y lo más sencillas posibles.
		\item Presentación breve de los problemas propuestos como ejercicio para la siguiente sesión.\\Los problemas propuestos son seleccionados entre los problemas disponibles en los jueces de programación en línea UVa \cite{UVa}, Codeforces \cite{Codeforces} y Spoj\cite{Spoj} buscando que se resuelvan utilizando los temas de la sesión y algunos temas de sesiones anteriores. Dado que el archivo de problemas en estos jueces es bastante extenso, la búsqueda de los problemas según los temas se hace con ayuda de los problemas que proponen Steven y Felix Halim en sus libros Competitive Programming 1/2 \cite{Halim, Halim2}, Ahmed Shamsul Arefin en su libro ``Art of Programming Contests''\cite{ArtOfProgramming} y con ayuda de dos buscadores de problemas en los cuales la búsqueda se hace de acuerdo al tema del cual trata el problema que son: el buscador de Codeforces y el un buscador de problemas para el sitio UVa desarrollado por Mark Greve \cite{UVAToolkit}. Luego de seleccionarlos, los problemas deben resolverse ya que sus soluciones no están disponibles en ninguno de los libros o sitios web mencionados anteriormente.
	\end{enumerate}

\section{Resultados}
% Se reportan los resultados, sin emitir juicios ni opiniones de carácter personal. Se deben definir los símbolos y las unidades empleadas, incluir figuras y tablas que reporten los resultados obtenidos. Los datos se presentan en forma  tal que otras personas los puedan usar, haciendo énfasis en el texto de aquellos aspectos que son más importantes en las tablas, gráficos o figuras. Dar barras de errores o límites de validez para datos numéricos o gráficos. Las tablas y figuras deben ser poseer un título que describa de manera concisa el contenido de lo que se cita. Igualmente, en este ítem se extraen los principios, relaciones, o generalizaciones de los resultados logradas como consecuencia del análisis de comportamiento de los mismos.

\subsection{Programa clase a clase}
El Semillero se desarrolló en 15 sesiones semanales de dos horas de duración cada una. Los contenidos trabajados en cada sesión se muestran en la tabla~\ref{Tabla:temas}. Acá se puede ver que la dificultad de los algoritmos va aumentando progresivamente y los elementos necesarios para entender e implementar un algoritmo específico se enseñan antes de dicho algoritmo. Ejemplos de esto serían la necesidad de aprender a utilizar los arreglos y los vectores antes de aprender las formas de representar un grafo, la importancia de conocer y entender el algoritmo de búsqueda en anchura y la estructura de datos del heap antes de implementar el algoritmo de Dijkstra, o explicar cómo funciona la programación dinámica antes de discutir el algoritmo de Floyd-Warshall.

\begin{table}
	\centering
	\begin{tabular}{|c|l|}
		\hline
		\textbf{Semana} & \textbf{Temas}\\
		\hline
		  & Introducción a C++ y a los jueces de programación \\
		1 & ¿En qué consiste una maratón de programación?\\
		  & Solución a un problema básico de maratón de programación\\
		\hline
		2 & Arreglos en C++, Vectores de C++ y Grafos\\
		\hline
		  & {\color{red}PONER TEMA JUAN OJO NO OLVIDAR} \\
		3 & Representación de grafos en C++\\
		  & Entrada usando \verb|getline| y \verb|stringstream|\\
		\hline
		  & Pila y Cola\\
		4 & Búsqueda en anchura (BFS)\\
		  & Búsqueda en profundidad (DFS)\\
		\hline
		5 & Problemas de BFS y DFS\\
		\hline
		  & Map, Set, Heap\\
		6 & Ordenamiento topológico\\
		  & Componentes fuertemente conexas (SCC)\\
		\hline
		7 & Algoritmo de Dijkstra \\
		\hline 
		8 & Algoritmo de Bellman-Ford\\
		\hline 
		9 & Programación dinámica: Problemas clásicos\\
		\hline
		10 & Algoritmo de Floyd-Warshall\\
		\hline
		11 & Árbol de mínima expansión\\
		\hline
		12 & Algoritmo de Knuth-Morris-Pratt\\
		\hline
		13 & Algoritmo de máximo flujo\\
		\hline
		14 & Solución de problemas de la IV Maratón de Programación UTP\\
		\hline
		15 & Algoritmos de teoría de números\\
		\hline 
	\end{tabular}
	
	\quad \\
	\caption{Temas desarrollados en el Semillero}
	\label{Tabla:temas}
\end{table}

\subsection{Documentación}
Con el fin de que facilitar la comprensión de los temas, de que los estudiantes tuvieran material con el cual pudieran repasar los temas de las secciones de manera independiente y de dejar un legado para las personas que vayan a estar a cargo del Semillero en el futuro, se decidieron crear diapositivas con los contenidos trabajados en cada sesión. El contenido de estas diapositivas se mantiene actualizado en el repositorio público \url{https://github.com/anaechavarria/SemilleroProgramacion/} y además se comparte con los estudiantes al final de cada sesión.\\
Durante el transcurso del semestre, se decidió dar a conocer el contenido de este repositorio con el director de las maratones de programación de la Universidad Tecnológica de Pereira (UTP) y el de la Universidad Pontificia Bolivariana (UPB) quienes difundieron la información entre sus estudiantes. Esta información llegó también a manos de estudiantes de la Universidad Sergio Arboleda, quienes escribieron para preguntar si ellos también podían hacer uso del material para prepararse para las maratones de programación. Luego de compartir la información con estas universidades, 8 personas nuevas empezaron a seguir el contenido del repositorio y otras 3 lo marcaron como favorito.

Adicional al material de las diapositivas, se desarrolló un manual con los algoritmos vistos en el Semillero en el transcurso del semestre (ver Anexo). Este manual se realizó pensando en que los estudiantes tuvieran documentados los algoritmos aprendidos durante el semestre y pudieran estudiarlos más fácilmente. Se espera que los estudiantes lo complementen con los temas que aprendan en el Semillero el próximo semestre y con otros temas que consideren de importancia, lo utilicen como su propio manual de algoritmos en las maratones de programación y les sirva como herramienta para solucionar los problemas que se les presenten en las competencias.

\subsection{Competencias}
Como se mencionó en la metodología, las competencias son parte fundamental del programa del Semillero. Es por esto que para cada sesión se buscaron y seleccionaron entre 2 y 4 problemas que tuvieran relación con los temas vistos en dicha sesión y se crearon competencias con dichos problemas en los sitios Contests: Factor Común\cite{FactorComun} y Virtual Online Contests\cite{AhmedAly}, dos páginas web para crear competencias con problemas de los jueces de programación mostrados en el Semillero.\\
Se crearon un total de 11 competencias, los resultados de cada competencia se muestran en la tabla~\ref{Tabla:competencias}. Allí se pueden observar que ocurrieron dos fenómenos importantes, el primero es que la cantidad de personas que participaron en el Semillero se redujo en el transcurso del semestre y el segundo es que en las competencias 6 a 9 el número de problemas resueltos es muy bajo. Estos fenómenos se deben a que a medida que avanza el semestre, los compromisos académicos son mayores lo que hace que se necesite más tiempo para las actividades la Universidad. Como el Semillero es de carácter opcional, muchos estudiantes se ven forzados a dejar de asistir para poder atender los compromisos académicos obligatorios o no tienen suficiente tiempo para resolver los problemas.

\begin{table}
	\centering
	\begin{tabular}{|c|c|l|c|}
		\hline
		\textbf{Competencia} & \textbf{Problemas} & \textbf{Participantes} & \textbf{Problemas resueltos} \\
		\hline
		  &   & agomezl  & 4\\
		  &   & svanegas & 4\\
		  &   & estebanf01 & 4\\
		1 & 4 & cmejia49 & 3\\
		  &   & yampyer & 3\\
		  &   & SantiSP & 2\\
		  &   & jlopera8 & 1\\
		  &   & srincon2 & 1\\
		\hline
		  &   & agomezl  & 2\\
		  &   & luisponce & 2\\
		  &   & svanegas & 1\\
		2 & 2 & estebanf01 & 0\\
		  &   & cmejia49 & 0\\
		  &   & yampyer & 0\\
		  &   & SantiSP & 0\\
		  &   & jlopera8 & 0\\
		\hline
		  &   & zubieta  & 4\\
		  &   & svanegas & 3\\
		  &   & luisponce & 1\\
		3 & 4 & jlopera8 & 1\\
		  &   & estebanf01 & 0\\
		  &   & cmejia49 & 0\\
		  &   & yampyer & 0\\
		\hline
		  &   & zubieta  & 3\\
		4 & 3 & svanegas & 2\\
		  &   & estebanf01 & 2\\
		  &   & luisponce & 2\\
		\hline 
		  &   & svanegas  & 3\\
		5 & 3 & luisponce & 3\\
		  &   & estebanf01 & 3\\
		  &   & zubieta & 1\\
		\hline
		6 & 4 & svanegas & 1\\
		  &   & spalac24 & 1\\
		\hline
		7 & 3 & yampyer & 0\\
		  &   & estebanf01 & 0\\
		\hline
		  &   & svanegas & 1\\
		8 & 3 & estebanf01 & 1\\
		  &   & yampyer & 0\\
		\hline
		9 & 2 & svanegas & 0\\
		  &   & estebanf01 & 0\\
		\hline
		10 & 2 & svanegas & 2\\
		   &   & estebanf01 & 1\\
		\hline
		11 & 4 & svanegas & Competencia actualmente\\
		   &   & estebanf01 & en ejecución\\
		\hline
	\end{tabular} 
	
	\quad\\
	\caption{Resultados de las competencias realizadas}
	\label{Tabla:competencias}
\end{table}

\subsection{Problemas resueltos}
Para el desarrollo del Semillero y de las competencias fue necesario buscar y solucionar problemas de los archivos de los jueces de programación que se ajustaran a los temas discutidos en casa sesión. Se resolvieron un total de 49 problemas; de estos 2 fueron resueltos en reuniones del Semillero, 34 fueron problemas propuestos para las competencias y los 13 restantes fueron los problemas de la IV Maratón de Programación UTP cuyas soluciones se discutieron en una sesión del Semillero. 

\subsection{Maratones de programación}
Como parte del objetivo del Semillero, se invitó a los estudiantes a participar en el Circuito Colombiano de Maratones de Programación. Estas son maratones que se realizan a nivel nacional y son preparatorias para la Maratón Nacional de Programación ACIS/REDIS que se realizará en octubre de este año. En dos de las cuatro competencias que se han realizado este año, se tuvo la participación de 2 equipos de estudiantes del Semillero \cite{CCMP}.\\
Por otro lado, el 4 de mayo se llevó a cabo la IV Maratón de Programación UTP que tenía una dificultad básica/intermedia y estaba pensada para el fortalecimiento de los competidores novatos \cite{UTP}. En esta competencia, un equipo conformado por dos estudiantes del Semillero quedó en cuarto lugar, compitiendo contra equipos de otras universidades de Colombia.\\



\section{Conclusión}
% Condensa los avances o aportes que en conocimiento dan los resultados obtenidos luego del desarrollo del proyecto.
El Semillero de Programación es un grupo importante de la Universidad en el que los estudiantes interesados en la programación y especialmente en competir en las maratones de programación tienen un espacio para aprender, discutir y socializar problemas. En los últimos años tres equipos de personas que han pertenecido al Semillero han clasificado a la Maratón Mundial de Programación ACM-ICPC pero actualmente los estudiantes del Semillero son novatos en las competencias y asisten a las reuniones buscando aprender y mejorar para poder tener buen desempeño en la Maratón Nacional de Programación ACIS/REDIS para poder clasificar a la Maratón Regional Suramericana ACM-ICPC.\\
El desarrollo de un plan de trabajo para el Semillero de Programación ha permitido que estos estudiantes mejoren sus conocimientos en estructuras de datos, el lenguaje de programación C++, los problemas clásicos de programación dinámica y en los principales algoritmos de grafos, teoría de números y strings. Por otro lado, la metodología utilizada para el desarrollo del Semillero, en la cual se hacen constantemente competencias, fomenta el trabajo individual de los estudiantes y fortalece sus habilidades de programación y de resolución de problemas, además de permitirles aplicar los conocimientos adquiridos en cada sesión.\\
Por otra parte, el plan de trabajo y el material utilizado en el Semillero (diapositivas, competencias, problemas y el manual de algoritmos) fueron desarrollados pensando en darle una estructura al Semillero de forma que los estudiantes se motiven a asistir por que los temas son adecuados para su nivel de conocimiento. Se busca que este contenido pueda ser utilizado por los estudiantes de la Universidad EAFIT y de otras universidades no solo para el estudio de manera independiente, sino también para que se siga desarrollando el Semillero y se implementen cursos similares en otras universidades.\\
Se pudo observar que el número de estudiantes que hacen parte del Semillero fue disminuyendo a medida que avanzaba el semestre. Esto se atribuye a que el curso no es de caracter obligatorio y los estudiantes, dado un aumento en sus compromisos académicos, se ven forzados a utilizar ese tiempo en las actividades obligatorias de la Universidad. Para evitar esto, se sugiere que el Semillero sea incorporado como un curso electivo del pénsum de Ingeniería de Sistemas e Ingeniería Matemática y de esta manera los estudiantes reciban créditos académicos por asistir a este curso y tengan un mayor compromiso con las actividades que allí se desarrollan. En universidades como la Universidad de los Andes y la Universidad Nacional de Singapur, se han implementado cursos electivos similares y han dado buenos resultados.\\
Durante el transcurso del semestre los estudiantes del Semillero han competido en maratones de programación a nivel nacional obteniendo el cuarto lugar en la IV Maratón de Programación UTP. Se espera que los estudiantes sigan mejorando y participando en este tipo de competencias para prepararse para la Maratón Nacional de Programación ACIS/REDIS que se realizará en octubre del presente año y poder clasificar a la Maratón Regional Suramericana de Programación ACM-ICPC y posiblemente a la Maratón Mundial de Programación ACM-ICPC.\\

\section{Agradecimientos}
% En este ítem se citan las personas o entidades gracias a las cuales pudo desarrollarse el proyecto gracias a sus contribuciones (financieras o en especie). Se deben señalar los nombres completos y las afiliaciones. Ejemplo: “I wish to thank Prof. L. M. Brown of the Cavendish Laboratory, Cambridge, for suggestion this review, and to acknowledge my debt to the books listed below”.
\begin{itemize}
	\item Al Departamento de Ingeniería de Sistemas de la Universidad EAFIT por su aporte financiero para la realización de este proyecto.
	\item A la Universidad EAFIT por proporcionar el espacio para las reuniones del Semillero.
	\item Al estudiante Santiago Palacio Gómez por su constante apoyo durante las sesiones del Semillero.
	\item Al profesor Francisco Correa por sus observaciones para la elaboración de los reportes.
	\item A los integrantes del Semillero de Programación por ser nuestra motivación para ser siempre mejores.
\end{itemize}

\nocite{*}
\bibliographystyle{IEEEtran}
\bibliography{Bibliography}
	% Se deben citar todas las fuentes de información de las cuales se hayan tomado datos útiles para la investigación. El formato más empleado presenta la referencia en el siguiente orden: Autores (apellido, inicial del primer nombre si son varios autores separados por coma igualmente); Título del artículo; Revista; Número; Volumen; páginas


\section*{Anexo}


\end{document}


