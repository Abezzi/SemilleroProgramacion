%%%%%%%%%%%%%%%%%%%%%%%%%%%%%%%%%%%%%%%%%%%%%%%%%%%%%%%%%%%%%%%%%%%%%%%%%%%
%
% Plantilla para un artculo en LaTeX en español.
%
%%%%%%%%%%%%%%%%%%%%%%%%%%%%%%%%%%%%%%%%%%%%%%%%%%%%%%%%%%%%%%%%%%%%%%%%%%%

\documentclass[11pt, oneside]{article}


% idioma
\usepackage[utf8]{inputenc}
\usepackage[spanish]{babel}

%tablas
\usepackage{booktabs}

%rotar tablas
\usepackage{rotating}

%color tablas
\usepackage{colortbl}

%espaciado
\usepackage{setspace}
\onehalfspacing
\setlength{\parindent}{0pt}
\setlength{\parskip}{2.0ex plus0.5ex minus0.2ex}


%margenes según n. icontec
\usepackage{vmargin}
\setmarginsrb           { 4.0cm}  % left margin
                        { 4.0cm}  % top margcm
                        { 2.0cm}  % right margcm
                        { 3.0cm}  % bottom margcm
                        {  10pt}  % head height
                        {0.25cm}  % head sep
                        {   9pt}  % foot height
                        { 0.3cm}  % foot sep


% inserción url's notas de pie.
\usepackage{url}
%  \usepackage{hyperref}


% Paquetes de la AMS:
\usepackage{amsmath, amsthm, amsfonts}

% Teoremas
%--------------------------------------------------------------------------
\newtheorem{thm}{Teorema}[section]
\newtheorem{cor}[thm]{Corolario}
\newtheorem{lem}[thm]{Lema}
\newtheorem{prop}[thm]{Proposición}
\theoremstyle{definition}
\newtheorem{defn}[thm]{Definición}
\theoremstyle{remark}
\newtheorem{rem}[thm]{Observación}

% Atajos.
% Se pueden definir comandos nuevos para acortar cosas que se usan
% frecuentemente. Como ejemplo, aqu se definen la R y la Z dobles que
% suelen representar a los conjuntos de nmeros reales y enteros.
%--------------------------------------------------------------------------

\def\RR{\mathbb{R}}
\def\ZZ{\mathbb{Z}}

% De la misma forma se pueden definir comandos con argumentos. Por
% ejemplo, aqu definimos un comando para escribir el valor absoluto
% de algo ms fcilmente.
%--------------------------------------------------------------------------
\newcommand{\abs}[1]{\left\vert#1\right\vert}

% Operadores.
% Los operadores nuevos deben definirse como tales para que aparezcan
% correctamente. Como ejemplo definimos en jacobiano:
%--------------------------------------------------------------------------
\DeclareMathOperator{\Jac}{Jac}

% 1.	TÍTULO
% Debe estar contenido en la portada del documento y reflejar en forma resumida, concisa y clara el objetivo de la propuesta de trabajo, sin que se escriba literalmente el objetivo propuesto.
% 
% 2.	TUTOR DEL PROYECTO
% Deberán indicarse, dentro de la portada del documento, los nombres y apellidos del tutor y en caso de haberlo también del co-tutor especificando la escuela y programa al que pertenecen y/o empresa que lo(s) designa (en el caso del co-tutor del sector industrial).

\newcommand\portada{
\begin{titlepage}
		\begin{center}
			{\large \bf Estudio del área geográfica apropiada para el desarrollo de un modelo de nicho ecológico del dengue en el Valle de Aburrá }
			\vfill
			{\large\bf Autor: \par}
			{\large Ana Echavarría Uribe\par}
			{\large\bf Tutor: \par}
			{\large Olga Lucía Quintero \par Universidad Eafit \par Escuela de Ciencias Básicas y Humanidades \par}
			{\large\bf Co-tutor: \par}
			{\large Sair Arboleda \par Universidad de Antioquia \par Instituto de Biología}
			\vfill
			{\large\bf Universidad Eafit  \par}
			{\large\bf Ingeniería Matemática \par}
			{\large\bf Medellín\par}
			{\large\bf 2012 \par}
		\end{center}
\end{titlepage}
}


\begin{document}
\portada


\renewcommand\contentsname{\centering Tabla de Contenidos}
\tableofcontents
\clearpage

\section{Planteamiento del Problema}
% Debe darse en términos del marco teórico matemático dentro del cual se halla el problema que se desea solucionar, indicando por qué y cómo puede ser resuelto desde las áreas de conocimiento que el estudiante domina.

La fiebre del dengue es una enfermedad viral transmitida por mosquitos luego de picar a una persona infectada. Esta enfermedad es endémica de los municipios de Medellín y Bello. A partir de datos geográficos, climáticos y poblacionales de la región se quiere, por medio de herramientas matemáticas y estadísticas, determinar el área más apropiada para la futura realización un modelo de predicción con el fin de poder determinar cuándo y dónde habrá epidemias del dengue.

\section{Objetivos generales y específicos}
% Se deben presentar el objetivo general y algunos específicos (tantos como sean suficientes para describir el tipo de resultados esperados). El objetivo general indica la meta global que se desea alcanzar, por lo general es bastante amplio, pero no intangible. Los objetivos específicos deben ser precisos y mostrar cómo se alcanzará la meta propuesta mediante pequeñas metas específicas, muy claras y tangibles.

\subsection{Objetivo General}
Determinar área geográfica apropiada para el desarrollo de un modelo de nicho ecológico del dengue en el Valle de Aburrá, teniendo en cuenta las condiciones para la formación de brote (aparición de la enfermedad) y criadero del mosquito transmisor del dengue.\\

\subsection{Objetivos Específicos}
\begin{itemize}
	\item{Crear una base de datos en tiempos de muestreo adecuados basándose en información satelital y datos estadísticos del IDEAM.}
	\item{Realizar análisis estadísticos como el análisis de correlación, media, varianza, covarianza y moda de los datos recopilados.}
	\item{Definir una estructura basada en teoría de sistemas lineales para crear hipótesis sobre las relaciones de causalidad entre las variables dando continuidad a la investigación de la tesis doctoral del co-tutor.}
	\item{Evaluar la influencia de factores socioeconómicos sobre el modelo de nicho ecológico del dengue si es posible a partir de la información disponible.}
\end{itemize}

\section{Antecedentes}
% Permiten identificar pasos previos realizados por otras personas o entidades con miras a la solución del problema objeto del proyecto o problemas similares, que pueden servir como punto de partida para la obtención de los resultados esperados dentro de la realización del proyecto. Los antecedentes resultan muy útiles si se describen en términos de trabajos previos realizados en el ámbito mundial, nacional y cuando lo amerite, local, en el tema específico relacionado con el proyecto propuesto.
El tema de esta práctica investigativa está basado en el la tesis doctoral del co-tutor en la cual se hace un mapeo del riesgo de transmisión de la fiebre de dengue en el Valle de Aburrá. Se busca dar continuación a dicha investigación adicionando la perspectiva de modelado matemático en el tratamiento de enfermedades epidemiolgógicas.

\section{Justificación}
% Describe las razones por las cuales es de interés académico, gubernamental  o empresarial resolver el problema de la propuesta. En este caso se debe enunciar claramente que áreas del conocimiento de la matemática se hacen necesarias para poder resolver el problema. Así mismo, es de gran importancia señalar el impacto que generará el desarrollo del proyecto en diversas áreas como social, ambiental, científico, económico, etc.
Este proyecto se hace con el fin de acercar el Instituto de Biología de la Universidad de Antioquia a los grupos de investigación del Departamento de Ciencias Básicas y Humanidades de la Universidad Eafit y en particular al grupo de investigación de modelado matemático para fomentar los lazos investigativos y tener una dinámica que permita avanzar en la formación de un programa de biología computacional. \\
Los resultados de este trabajo servirán también para fundamentar la realización de un modelo de predicción espacio-temporal del dengue en el Valle de Aburrá y poder tomar medidas anticipadas para prevenirlas.

\section{Alcance}
% Debe identificarse todo aquello que se puede hacer en torno a la solución del problema planteado lo cual delimita el dominio del mismo. Una vez identificado dicho dominio, debe establecerse qué partes del todo se abordarán en el presente proyecto indicando las razones por las cuales no se tomarán en cuenta los demás aspectos del dominio global del problema. 
Los resultados de este proyecto, de ser exitosos, permitirán fundamentar la realización de un modelo de predicción  espacio-temporal de epidemias del dengue en el Valle de Aburrá.

\section{Metodología Propuesta}
% Dentro de la metodología deben describirse detalladamente en qué consistirán las actividades o etapas globales del proyecto, tipo de información o datos requeridos, variables a considerar, tipo de medidas a realizar, cantidad, número de muestras y equipos que se utilizarán y normativa requerida que se aplicará. En los casos que lo ameriten, debe incluirse el diseño de experimentos a seguir (variables a medir o controlar, medidas a realizar y número de muestras requeridas).
Para el desarrollo del proyecto se emplearán los datos obtenidos por el co-tutor para el desarrollo de su tesis doctoral. Luego de construir la base de datos, se harán análisis estadísticos y modelos matemáticos utilizando Matlab, la teoría de sistemas lineales y la estadística.

\section{Cronograma de Actividades}
% Una vez establecida la metodología a seguir, es posible establecer una lista de actividades que conducirán al logro de los objetivos planteados. Puede presentarse el cronograma en forma de tabla que liste cada actividad y el tiempo que se dedicará a desarrollar cada una en meses o preferiblemente en semanas.
El tiempo de trabajo de la práctica será de 5 semanas con una intensidad horaria semanal de 7 horas. \\ \quad \\
\begin{tabular}{|c|c|}
	\hline
	\textbf{Actividad} & \textbf{Duración} \\
	\hline \hline
	Creación de la base de datos & 1.5 semanas \\
	\hline
	Realización de los análisis estadísticos & 1 semana \\
	\hline
	Creación del modelo matemático lineal & 1.5 semanas \\
	\hline
	Análisis de resultados y trabajos complementarios & 1 semana \\
	\hline
\end{tabular}

%\section{Presupuesto}
%¿SOFTWARE?
% Este ítem, en muchos casos no será necesario, pero en caso de que el proyecto lo amerite es importante cuantificar los costos de los siguientes aspectos, los cuales se pueden presentar en una tabla por cantidades unitarias y totales:

% •	Personal (horas del personal que interviene en el desarrollo de la propuesta, con su costo individual por hora y total según la cantidad de horas de dedicación al proyecto).
% •	Materiales (dentro de los cuales se pueden incluir insumos de laboratorio requeridos, materiales de oficina, muestras, etc). Estos materiales deben cuantificarse aún en caso de que no deban comprarse y posteriormente se puede señalar si se obtienen en especie o por donación de alguna institución, grupo de investigación o persona en particular.
% •	Uso de equipos (debe cuantificarse el costo que por uso de equipos tendría el proyecto, aún cuando no sea necesario pagarlo, pues permite cuantificar cuánto habría que pagar en caso de requerirlo).
% •	Servicios (se refiere a servicios externos a las actividades del estudiante que podría ser necesario contratar para lograr el objetivo propuesto, por ejemplo, ensayos de laboratorio en instituciones ajenas a la Universidad, procesos de fabricación, etc.).
% •	Otros (deben señalarse otros costos en los que se requiera incurrir para la realización del proyecto y que no hayan sido considerados en los ítems anteriores, debe señalarse el concepto por el cual se daría el costo).
% Una vez cuantificados los costos de los anteriores aspectos es fundamental señalar el responsable de cubrir cada ítem, que podría ser: la Universidad a través del Centro de Laboratorios, Centro de costos de algún proyecto de investigación en particular, el estudiante, la empresa financiadora, o cualquier otra fuente de financiación que aporte los recursos económicos o en especie para la ejecución del proyecto.

\section{Propiedad Intelectual}
% Es de gran importancia señalar a quién pertenecerán los derechos de propiedad intelectual de los resultados a obtener, bajo qué concepto (derechos de autor, patente, registro, etc) y en qué proporción en caso de que le perteneciesen a más de un propietario, señalando las razones de acuerdo de dicha propiedad (por ejemplo, por inversión, por dedicación, etc). Se sugiere consultar la documentación existente en la Universidad %(“Reglamento de propiedad intelectual” disponible en http://www.eafit.edu.co/institucional/reglamentos/Documents/Reglamento_Propiedad_Intelectual.pdf )
Los resultados del trabajo pertenecen en un 50$\%$ a la estudiante Ana Echavarría, en un 25$\%$ al tutor y otro 25$\%$ al co-tutor.



\section{Bibliografía}
% Se referencia en este ítem los documentos consultados para la elaboración de la propuesta, los cuales pueden ser de carácter científico (libros, artículos de revistas científicas, páginas web) o de otro tipo (reglamento de propiedad intelectual, norma ICONTEC para elaboración de documentos escritos, legislación sobre propiedad intelectual en el país, etc).
ARBOLEDA SÁNCHEZ, Sair. Mapeo del riesgo de transmisión de fiebre por dengue en un área endémica de Colombia. Medellín, 2011, 169p. Tesis doctoral (Biología). Universidad de Antioquia. Facultad de Ciencias Exactas y Naturales. \\ \quad \\
NASA. The Landsat program [en línea]. Disponible en http://landsat.gsfc.nasa.gov/


\end{document}