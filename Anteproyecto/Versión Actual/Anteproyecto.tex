\documentclass[11pt, oneside]{article}
% idioma
\usepackage[utf8]{inputenc} % Tildes
\usepackage[spanish]{babel} % Separación en español
\usepackage{booktabs} % Tablas
\usepackage{rotating} % Rotar Tablas
\usepackage{colortbl} % Color Tablas
\usepackage{url} % Inserta URL's de las notas al pie
\usepackage{hyperref} % Crea accesos directos dentro del documento
\usepackage{amsmath, amsthm, amsfonts} % Paquete de la AMS
\usepackage{longtable} % Para que una tabla se parta por la página
	
\usepackage{setspace} % Espaciado
	\onehalfspacing
	\setlength{\parindent}{0pt}
	\setlength{\parskip}{2.0ex plus0.5ex minus0.2ex}
\usepackage{vmargin} % Márgenes Icontec
	\setmarginsrb   { 4.0cm}  % left margin
	                { 4.0cm}  % top margcm
	                { 2.0cm}  % right margcm
	                { 3.0cm}  % bottom margcm
	                {  10pt}  % head height
	                {0.25cm}  % head sep
	                {   9pt}  % foot height
	                { 0.3cm}  % foot sep

% Teoremas
%--------------------------------------------------------------------------
\newtheorem{thm}{Teorema}[section]
\newtheorem{cor}[thm]{Corolario}
\newtheorem{lem}[thm]{Lema}
\newtheorem{prop}[thm]{Proposición}
\theoremstyle{definition}
\newtheorem{defn}[thm]{Definición}
\theoremstyle{remark}
\newtheorem{rem}[thm]{Observación}

% Atajos.
% Se pueden definir comandos nuevos para acortar cosas que se usan
% frecuentemente. Como ejemplo, aqu se definen la R y la Z dobles que
% suelen representar a los conjuntos de nmeros reales y enteros.
%--------------------------------------------------------------------------
\def\RR{\mathbb{R}}
\def\ZZ{\mathbb{Z}}
\newcommand{\abs}[1]{\left\vert#1\right\vert}
\addto\captionsspanish{\renewcommand{\refname}{Bibliografía}}


% 1.	TÍTULO
% Debe estar contenido en la portada del documento y reflejar en forma resumida, concisa y clara el objetivo de la propuesta de trabajo, sin que se escriba literalmente el objetivo propuesto.
% 
% 2.	TUTOR DEL PROYECTO
% Deberán indicarse, dentro de la portada del documento, los nombres y apellidos del tutor y en caso de haberlo también del co-tutor especificando la escuela y programa al que pertenecen y/o empresa que lo(s) designa (en el caso del co-tutor del sector industrial).

\newcommand\portada{
\begin{titlepage}
		\begin{center}
			{\large \bf Desarrollo e implementación de un programa de trabajo para el semillero de programación}
			\vfill
			{\large\bf Autor: \par}
			{\large Ana Echavarría Uribe\par}
			{\large\bf Tutor: \par}
			{\large Juan Francisco Cardona Mc'Cormick \par Universidad Eafit \par Escuela de Ingeniería \par fcardona@eafit.edu.co }
			\vfill
			{\large\bf Universidad Eafit  \par}
			{\large\bf Ingeniería Matemática \par}
			{\large\bf Medellín\par}
			{\large\bf 2013 \par}
		\end{center}
\end{titlepage}
}


\begin{document}
\portada


\renewcommand\contentsname{\centering Tabla de contenidos}
\tableofcontents
\clearpage

% 3.	PLANTEAMIENTO DEL PROBLEMA
% Debe darse en términos del marco teórico matemático dentro del cual se halla el problema que se desea solucionar, indicando por qué y cómo puede ser resuelto desde las áreas de conocimiento que el estudiante domina.
\section{Planteamiento del problema}
El semillero de programación es un grupo de la Universidad que tiene como objetivo presentar nuevas técnicas de programación a estudiantes interesados en mejorar sus habilidades en esta área y prepararlos para participar en las maratones de programación realizadas por ACIS/REDIS \cite{ACIS} y por la ACM-ICPC \cite{ICPC}. En este semillero se discuten los temas más útiles \cite{ProgrammingChallenges, Halim, Halim2, Brasil} para estas competencias que son los algoritmos de grafos, strings y teoría de números, programación dinámica, la recursividad y las estructuras de datos.\\
Durante los últimos años el semillero ha estado a cargo de estudiantes destacados en las maratones de programación bajo la supervisión de docentes del departamento de ingeniería de sistemas; sin embargo, durante este tiempo no se desarrolló un plan de trabajo para el semillero. Lo que se busca en esta práctica investigativa es desarrollar e implementar un programa de trabajo para el semillero de programación que le dé a los estudiantes las bases teóricas y prácticas para mejorar sus habilidades y enfrentarse por primera vez a las maratones de programación.

\section{Objetivos generales y específicos}
% 4.	OBJETIVOS GENERAL Y ESPECÍFICOS
% Se deben presentar el objetivo general y algunos específicos (tantos como sean suficientes para describir el tipo de resultados esperados). El objetivo general indica la meta global que se desea alcanzar, por lo general es bastante amplio, pero no intangible. Los objetivos específicos deben ser precisos y mostrar cómo se alcanzará la meta propuesta mediante pequeñas metas específicas, muy claras y tangibles.

\subsection{Objetivo general}
Desarrollar e implementar un plan de actividades y temas de fundamentación matemática para el semillero de programación que busque mejorar las habilidades de programación de los estudiantes, basándose en la metodología usada por Steven Halim en el curso Competitive Programming de la Universidad Nacional de Singapur y en otras que en el desarrollo del proyecto puedan surgir.

\subsection{Objetivos específicos}
\begin{itemize}
	\item{Adaptar, al nivel de los estudiantes del semillero de programación, la metodología de clases del curso Competitive Programming dictado por Steven Halim \cite{CourseNUS} y otras que se puedan ser de utilidad.}
	\item{Presentar y documentar los conceptos básicos de la STL (Standard Template Library) de C++.}
	\item{Introducir a los estudiantes a algunos jueces en línea como \emph{Codeforces} \cite{Codeforces}, \emph{UVa} \cite{UVa} y \emph{Spoj} \cite{Spoj}.}
	\item{Buscar, resolver y sugerir problemas en los diferentes jueces que permitan a los estudiantes aplicar los conceptos aprendidos en las reuniones y queden como banco de problemas para uso futuro.}
	\item{Hacer revisiones de las soluciones a los problemas propuestos luego de que los estudiantes hayan intentado resolverlos de manera independiente, discutiendo los algoritmos y fundamentos matemáticos involucrados.}
	\item{Crear material de trabajo (diapositivas, programas, problemas, competencias) con los temas enseñados y compartirlo con los estudiantes para que les sirvan como material de estudio en casa.}
	\item{Crear el repositorio \emph{github.com/anaechavarria/SemilleroProgramacion} y actualizarlo cada semana con el material del semillero.}
	\item{Presentar y documentar los algoritmos de recorrido de grafos en profundidad (DFS) y en anchura (BFS) y sus aplicaciones.}
	\item{Presentar y documentar los algoritmos para hallar los caminos más cortos en grafos: Dijkstra, Bellman-Ford y Floyd-Warshall.}
	\item{Presentar y documentar en qué consisten los métodos y técnicas de programación dinámica y los problemas más comunes que se resuelven usándolos.}
	\item{Mostrar cómo se pueden hallar las ocurrencias de una palabra en un texto de manera rápida usando el algoritmo de KMP.}
	\item{Presentar y documentar los algoritmos de teoría de números más útiles \cite{ProgrammingChallenges, Halim, Halim2, Brasil} para las competencias de programación.}
	\item{Presentar y documentar el algoritmo de Kruskal para hallar el árbol de mínima expansión.}
	\item{Presentar y documentar los algoritmos de geometría más usados \cite{ProgrammingChallenges, Halim, Halim2, Brasil} en las competencias de programación.}
	\item{Promover la participación de los estudiantes en las maratones de programación.}
	\item{Dejar un legado estructurado y organizado para generaciones futuras con fundamentos matemáticos autocontenidos y estructurados.}
\end{itemize}

\section{Antecedentes}
% 5.	ANTECEDENTES
% Permiten identificar pasos previos realizados por otras personas o entidades con miras a la solución del problema objeto del proyecto o problemas similares, que pueden servir como punto de partida para la obtención de los resultados esperados dentro de la realización del proyecto. Los antecedentes resultan muy útiles si se describen en términos de trabajos previos realizados en el ámbito mundial, nacional y cuando lo amerite, local, en el tema específico relacionado con el proyecto propuesto.
El semillero de programación fue establecido hace varios años en la Universidad EAFIT y ha permitido que los estudiantes con interés en la programación y en las matemáticas tengan un espacio para aprender, interactuar y socializar problemas. En los años anteriores, el semillero era un espacio muy productivo para los estudiantes avanzados ya que durante las reuniones se socializaban y resolvían problemas de dificultad media y alta lo que le permitía, a quienes asistían, mejorar su habilidad de solución de problemas. Para los principiantes, sin embargo, el nivel del semillero era avanzado y se trataban problemas que ellos todavía no tenían la fundamentación teórica para resolver.\\
El semestre 2012-1 se dividió el semillero de programación en dos grupos: básico y avanzado. En el semillero avanzado se siguieron resolviendo y discutiendo problemas como se hacía anteriormente y el semillero básico se enfocó en desarrollar las habilidades de programación de los estudiantes de primero y segundo semestre. Este cambio logró despertar el interés de nuevos alumnos que anteriormente no asistían al semillero porque el nivel era muy avanzado para ellos.\\
Similar a lo que se ha trabajado en el semillero de programación existe desde 2008 un curso en la Universidad Nacional de Singapur (NUS) llamado Competitive Programming que busca fortalecer las habilidades de programación de sus estudiantes destacados para prepararlos para las competencias universitarias de programación de la ICPC \cite{PaperHalim}. Este curso está enfocado para estudiantes de tercer año de las áreas de ciencias de la computación, matemáticas e ingeniería electrónica que sean destacados en programación y busca enseñar algoritmos avanzados de manera práctica en lugar de demostrar la corrección de los algoritmos de manera rigurosa.\\
El curso de la NUS está basado en los libros \cite{Halim, Halim2} escritos por Steven Halim donde se explican los algoritmos y sus implementaciones y luego se propone una lista de problemas que se resuelven utilizando esos algoritmos. Por lo anterior, la metodología de este curso consta tanto de clases magistrales como de competencias ya que Steven Halim considera que es parte de la naturaleza humana tratar de lograr los mejores resultados cuando se compite con los demás \cite{PaperHalim}. Para complementar el proceso, se realizará una búsqueda de experiencias similares y se hará contacto con los autores.

\section{Justificación}
% 6.	JUSTIFICACIÓN
% Describe las razones por las cuales es de interés académico, gubernamental  o empresarial resolver el problema de la propuesta. En este caso se debe enunciar claramente que áreas del conocimiento de la matemática se hacen necesarias para poder resolver el problema. Así mismo, es de gran importancia señalar el impacto que generará el desarrollo del proyecto en diversas áreas como social, ambiental, científico, económico, etc.
Actualmente, los estudiantes que pertenecen al semillero de programación son en su mayoría de tercer semestre, teniendo también estudiantes de primer semestre. Lo anterior quiere decir que tienen conocimientos acerca de cómo programar mas no conocen las técnicas más utilizadas en la solución de problemas de maratones de programación como: los algoritmos de grafos, strings y geometría, la programación dinámica y los conceptos y algoritmos básicos de teoría de números. Este semestre se busca utilizar el programa de clases que desarrolló Steven Halim y otras metodologías que puedan surgir y adaptarlas para que se ajuste al nivel de los estudiantes del semillero, de manera que tengan la fundamentación teórica necesaria para resolver los problemas más comunes presentados en las maratones de programación.\\
Por otro lado se busca que la documentación y diapositivas utilizadas este semestre sirvan como material de trabajo para futuras generaciones del semillero. Por otro lado, el plan de trabajo permitirá que aquellos que continúen a cargo del semillero de programación el semestre entrante sepan cuáles son los conceptos que los estudiantes actuales conocen y puedan continuar enseñándoles aquellos temas que los estudiantes todavía no conocen.\\
El curso de la NUS de Competitive Programming, dictado por Steven Halim, comenzó en agosto de 2008 y muchos de sus estudiantes han logrado resultados favorables en las competencias de la ICPC, obteniendo incluso mención honorífica en Maratón Mundial ACM-ICPC celebrada en Varsovia en 2012 \cite{AlgoNUS}. Al igual que en este curso, se espera que los temas enseñados a los estudiantes durante este semestre y el próximo sirvan para que ellos tengan un buen desempeño en la Maratón Nacional de Programación ACIS/REDIS que se realiza en octubre y tengan la posibilidad de participar en la Maratón Regional Suramericana ACM-ICPC y en la Maratón Mundial ACM-ICPC bajo el nombre la Universidad, como ha ocurrido en ocasiones anteriores.

\section{Alcance}
% 7.	ALCANCE
% Debe identificarse todo aquello que se puede hacer en torno a la solución del problema planteado lo cual delimita el dominio del mismo. Una vez identificado dicho dominio, debe establecerse qué partes del todo se abordarán en el presente proyecto indicando las razones por las cuales no se tomarán en cuenta los demás aspectos del dominio global del problema.
La implementación de este proyecto permitirá que los estudiantes participantes en el semillero de programación, en su mayoría pertenecientes a las carreras de Ingeniería de Sistemas e Ingeniería Matemática de la Universidad se motiven y estén preparados para participar en el Circuito de Maratones de Programación ACIS/REDIS y en la Maratón Nacional de Programación ACIS/REDIS. De obtener buenos resultados en esta última competencia, los estudiantes podrían tener la oportunidad participar bajo el nombre de la Universidad en la Maratón Regional Suramericana ACM-ICPC y en la Maratón Mundial ACM-ICPC.

\section{Metodología propuesta}
% 8.	METODOLOGÍA PROPUESTA
% Dentro de la metodología deben describirse detalladamente en qué consistirán las actividades o etapas globales del proyecto, tipo de información o datos requeridos, variables a considerar, tipo de medidas a realizar, cantidad, número de muestras y equipos que se utilizarán y normativa requerida que se aplicará. En los casos que lo ameriten, debe incluirse el diseño de experimentos a seguir (variables a medir o controlar, medidas a realizar y número de muestras requeridas).
El semillero de programación consiste en una reunión semanal que dura entre 1 hora y media y 2 horas. La estructura de cada reunión se compone de tres etapas principales:
\begin{enumerate}
	\item{Discusión, solución y revisión de los problemas propuestos como tarea en la sesión anterior.}
	\item{Exposición del nuevo tema a trabajar, mostrando los algoritmos y elementos matemáticos relacionados.}
	\item{Presentación breve de los problemas propuestos como ejercicio para la siguiente sesión.}
\end{enumerate}
Para desarrollar cada reunión es necesario investigar acerca del tema a enseñar, preparar documentación (diapositivas, programas, competencias) que le permita a los estudiantes repasar lo aprendido y buscar y solucionar problemas que se ajusten al nivel y los temas aprendidos en la clase.

\section{Cronograma de actividades}
% 9.	CRONOGRAMA DE ACTIVIDADES
% Una vez establecida la metodología a seguir, es posible establecer una lista de actividades que conducirán al logro de los objetivos planteados. Puede presentarse el cronograma en forma de tabla que liste cada actividad y el tiempo que se dedicará a desarrollar cada una en meses o preferiblemente en semanas.
El semillero consistirá de 15 reuniones que se celebrarán todos los viernes desde el 1 de febrero hasta el 17 de mayo de 2013. Los temas a tratar en cada reunión son: \\ 

\begin{longtable}{|l|l|}
	\hline
	\textbf{Fecha} & \textbf{Tema} \\
	\hline \hline
	Febrero 1 & Introducción a C++ y al juez de programación Codeforces \cite{Codeforces}\\
	          & ¿En qué consiste una maratón de programación?\\
			  & Solución a un problema básico de maratón de programación  \qquad \qquad \qquad \qquad\\
	\hline
	Febrero 8 & Arreglos en C++, Vectores de C++ y Grafos\\
	\hline
	Febrero 15 & STL de C++\\
	\hline
	Febrero 22 & Algoritmos de recorrido de grafos: BFS y DFS\\
	\hline
	Marzo 1 & Aplicaciones de los algoritmos de BFS y DFS\\
	\hline
	Marzo 8 & Ordenamiento Topológico y Componentes Fuertemente Conexas\\
	\hline
	Marzo 15 & Algoritmos de Dijkstra y Bellman-Ford\\
	\hline
	Marzo 22 & ¿Qué es la programación dinámica?\\
	         & Soluciones básicas con programación dinámica\\
	         & Algoritmo de Floyd-Warshall\\
	\hline
	Abril 5 & Aplicaciones de la programación dinámica:\\
	        & Longest Common Substring (LCS) y 0-1 Knapsack\\
	\hline
	Abril 12 & Algoritmo de Knuth-Morris-Pratt\\
	\hline
	Abril 19 & Algoritmos básicos de teoría de números\\
	\hline
	Abril 26 & Solución de problemas de repaso\\
	\hline
	Mayo 3 & Algoritmo de Kruskal para hallar el árbol de mínima expansión (MST)\\
	\hline
	Mayo 10 & Algoritmo de Edmonds-Karp para hallar el máximo flujo\\
	\hline
	Mayo 17 & Algoritmos de geometría\\
	\hline
\end{longtable}

\section{Presupuesto}
% 10.	PRESUPUESTO
% Este ítem, en muchos casos no será necesario, pero en caso de que el proyecto lo amerite es importante cuantificar los costos de los siguientes aspectos, los cuales se pueden presentar en una tabla por cantidades unitarias y totales:
% •	Personal (horas del personal que interviene en el desarrollo de la propuesta, con su costo individual por hora y total según la cantidad de horas de dedicación al proyecto).
% •	Materiales (dentro de los cuales se pueden incluir insumos de laboratorio requeridos, materiales de oficina, muestras, etc). Estos materiales deben cuantificarse aún en caso de que no deban comprarse y posteriormente se puede señalar si se obtienen en especie o por donación de alguna institución, grupo de investigación o persona en particular.
% •	Uso de equipos (debe cuantificarse el costo que por uso de equipos tendría el proyecto, aún cuando no sea necesario pagarlo, pues permite cuantificar cuánto habría que pagar en caso de requerirlo).
% •	Servicios (se refiere a servicios externos a las actividades del estudiante que podría ser necesario contratar para lograr el objetivo propuesto, por ejemplo, ensayos de laboratorio en instituciones ajenas a la Universidad, procesos de fabricación, etc.).
% •	Otros (deben señalarse otros costos en los que se requiera incurrir para la realización del proyecto y que no hayan sido considerados en los ítems anteriores, debe señalarse el concepto por el cual se daría el costo).
% Una vez cuantificados los costos de los anteriores aspectos es fundamental señalar el responsable de cubrir cada ítem, que podría ser: la Universidad a través del Centro de Laboratorios, Centro de costos de algún proyecto de investigación en particular, el estudiante, la empresa financiadora, o cualquier otra fuente de financiación que aporte los recursos económicos o en especie para la ejecución del proyecto.
	\subsection{Personal}
	La dedicación de la estudiante Ana Echavarría Uribe al proyecto será de 10 horas a la semana entre el 28 de enero y el 25 de mayo de 2013. Estas horas serán remuneradas al valor que tenga establecido la Universidad EAFIT para las monitorías y el costo será asumido por el Departamento de Ingeniería de Sistemas.
	\subsection{Materiales}
	Para la realización del proyecto será necesario el uso de un aula de clases cada semana durante dos horas. Esta aula será proporcionada por la Universidad EAFIT.


\section{Propiedad intelectual}
% 11.	PROPIEDAD INTELECTUAL
% Es de gran importancia señalar a quién pertenecerán los derechos de propiedad intelectual de los resultados a obtener, bajo qué concepto (derechos de autor, patente, registro, etc) y en qué proporción en caso de que le perteneciesen a más de un propietario, señalando las razones de acuerdo de dicha propiedad (por ejemplo, por inversión, por dedicación, etc). Se sugiere consultar la documentación existente en la Universidad (“Reglamento de propiedad intelectual” disponible en http://www.eafit.edu.co/institucional/reglamentos/Documents/Reglamento_Propiedad_Intelectual.pdf )
Debido a que este trabajo es una actividad continua de la Universidad, el producto de este trabajo pertenece le en un 100 $\%$. El contenido del trabajo podrá ser utilizado en su totalidad para el desarrollo del semillero de programación de la Universidad sin previo aviso a los autores.

\section{ Glosario}

\begin{description}
	\item[ACIS/REDIS] Alianza entre la Asociación Colombiana de Ingenieros de Sistemas (ACIS) y la Red de Decanos y Directores de Ingeniería de Sistemas y Afines (REDIS) que organiza actividades alrededor de Maratones de Programación en Colombia y Suramérica \cite{ACIS}.
	\item[ACM-ICPC] (International Collegiate Programming Contest) es la entidad organizadora de los concursos internacionales de programación para universitarios \cite{ICPC}.
	\item[STL] (Standard Template Library) es una librería de C++ que provee contenedores, funciones y algoritmos que pueden ser utilizadas con cualquier tipo de dato \cite{C++}.
	\item[Juez en línea] Sistema en línea que hace un juzgamiento o evaluación automática de un conjunto de problemas de programación que se encuentren en su base de datos.
	\item[Revisión] Discusión de un código por parte de un grupo de programadores con el fin de aprender las técnicas que se implementan allí.
	\item[Repositorio] Estructura de datos, almacenada en un servidor, que contiene, entre otras cosas, un conjunto de archivos y el historial de los cambios hechos a estos archivos.
	\item[BFS] (Breath-First Search) es un algoritmo de búsqueda de grafos que hace un recorrido en anchura, es decir, visita todos los nodos a distancia $k$ del nodo de inicio antes de visitar cualquier nodo a distancia $k+1$ \cite{Cormen}.
	\item[DFS] (Depth-First Search) es un algoritmo de búsqueda de grafos que hace un recorrido en profundidad, es decir, se visitan todos los nodos salientes del nodo visto más recientemente antes de visitarlo a él \cite{Cormen}.
	\item[KMP] Algoritmo desarrollado por Knuth, Morris y Pratt que hace una búsqueda de una palabra en un texto en tiempo lineal \cite{Cormen}.
\end{description}



% 12.	BIBLIOGRAFÍA
% Se referencia en este ítem los documentos consultados para la elaboración de la propuesta, los cuales pueden ser de carácter científico (libros, artículos de revistas científicas, páginas web) o de otro tipo (reglamento de propiedad intelectual, norma ICONTEC para elaboración de documentos escritos, legislación sobre propiedad intelectual en el país, etc).

\bibliographystyle{plain}
\bibliography{Biblio}

\end{document}
