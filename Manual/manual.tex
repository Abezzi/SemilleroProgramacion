\documentclass[10pt,letterpaper,twocolumn]{article}
\usepackage[utf8]{inputenc}
\usepackage[spanish]{babel}
\usepackage{listings}
\usepackage[usenames,dvipsnames]{color}
\usepackage{amsmath}
\usepackage{calc}
\usepackage{verbatim}
\usepackage{hyperref}
\usepackage{color}
\usepackage{geometry}
\usepackage[pdftex]{graphicx}
\usepackage{pgfplots}
\usepackage{array}

%Poner la página en landscape
\geometry{verbose,landscape,letterpaper,tmargin=1.5cm,bmargin=2cm,lmargin=1.6cm,rmargin=1.6cm}

\newcommand{\source}[1]{
	\verbatiminput{#1}
	\dotfill
}
\setlength{\columnsep}{0.25in}
\setlength{\columnseprule}{1px}


\begin{document}


\title{Manual de algoritmos del semillero de programación EAFIT}
\author{Ana Echavarría}
\date{\today}
\maketitle

\tableofcontents

\section{Plantilla}
	\source{./src/template.cpp}

\section{Grafos}

	\subsection{BFS}
		Algoritmo de recorrido de grafos en anchura que empieza desde una fuente $s$ y visita todos los nodos alcanzables desde $s$.\\
		El BFS también halla la distancia más corta entre $s$ y los demás nodos si las aristas tienen todas peso 1.\\
		Complejidad: $O(n+m)$ donde $n$ es el número de nodos y $m$ es el número de aristas.\\
		\source{./src/bfs.cpp}
		
	\subsection{DFS}
		Algoritmo de recorrido de grafos en profundidad que empieza visita todos los nodos del grafo.\\
		El algoritmo puede ser modificado para que retorne información de los nodos según la necesidad del problema.\\
		El grafo tiene un ciclo $\leftrightarrow$ si en algún momento se llega a un nodo marcado como gris.\\
		Complejidad: $O(n+m)$ donde $n$ es el número de nodos y $m$ es el número de aristas.\\
		\source{./src/dfs.cpp}
		
	\subsection{Ordenamiento topológico}
		Dado un grafo no cíclico y dirigido (DAG), ordena los nodos linealmente de tal forma que si existe una arista entre los nodos $u$ y $v$ entonces $u$ aparece antes que $v$ en el ordenamiento.\\
		Este ordenamiento se puede ver como una forma de poner todos los nodos en una línea recta y que las aristas vayan todas de izquierda a derecha.\\
		Complejidad: $O(n+m)$ donde $n$ es el número de nodos y $m$ es el número de aristas.\\
		\source{./src/ordenamiento_topologico.cpp}
		
	\subsection{Componentes fuertemente conexas}
		Dado un grafo dirigido, calcula la componente fuertemente conexa (SCC) a la que pertenece cada nodo.\\
		Para cada pareja de nodos $u, v$ que pertenecen a una misma SCC se cumple que hay un camino de $u$ a $v$ y de $v$ a $u$.\\
		Si se comprime el grafo dejando como nodos cada una de las componentes se quedará con un DAG.\\
		Complejidad: $O(n+m)$ donde $n$ es el número de nodos y $m$ es el número de aristas.\\
		\source{./src/componentes_conexas.cpp}
	
	\subsection{Algoritmo de Dijkstra}
		Dado un grafo con pesos \textbf{no negativos} en las aristas, halla la mínima distancia entre una fuente $s$ y los demás nodos.\\
		Al heap se inserta primero la distancia y luego en nodo al que se llega. Si se quieren modificar los pesos por \verb|long long| o por \verb|double| se debe cambiar en los tipos de dato \verb|dist_node| y \verb|edge|.\\
		Complejidad: $O((n+m) \operatorname{log} n)$ donde $n$ es el número de nodos y $m$ es el número de aristas.\\
		\source{./src/dijkstra.cpp}
		
	\subsection{Algoritmo de Bellman-Ford}
		Dado un grafo con pesos cualquiera, halla la mínima distancia entre una fuente $s$ y los demás nodos.\\
		Si hay un ciclo de peso negativo en el grafo, el algoritmo lo indica.\\
		Complejidad: $O(n \times m)$ donde $n$ es el número de nodos y $m$ es el número de aristas.\\
		\source{./src/bellman_ford.cpp}
	
	\subsection{Algoritmo de Floyd-Warshall}
		Dado un grafo con pesos cualquiera, halla la mínima distancia entre cualquier para de nodos.\\
		Si este algoritmo es muy lento para el problema ejecutar $n$ veces el algoritmo de Dijkstra o de Bellman-Ford según el caso.\\
		Complejidad: $O(n^3)$ donde $n$ es el número de nodos.\\ \quad \\
		Casos base: 
		$\verb|d[i][j]| =
			\begin{cases}
				0       &\text{\quad si } i = j\\
				w_{i,j} &\text{\quad si existe una arista entre $i$ y $j$}\\
				+\infty &\text{\quad en otro caso}
			\end{cases}$ \\
			
		\quad \\ \textbf{Nota:} Utilizar el tipo de dato apropiado (\verb|int|, \verb|long long|, \verb|double|) para \verb|d| y para $+\infty$ según el problema.
		
		\source{./src/floyd-warshall.cpp}
		
		\subsubsection{Clausura transitiva}
		Dado un grafo cualquiera, hallar si existe un camino desde $i$ hasta $j$ para cualquier pareja de nodos $i, j$\\ \quad \\
		Casos base:
		$\verb|d[i][j]| =
			\begin{cases}
				\verb|true |   &\text{\quad si } i = j\\
				\verb|true |   &\text{\quad si existe una arista entre $i$ y $j$}\\
				\verb|false|   &\text{\quad en otro caso}
			\end{cases}$ \\ \quad \\
		Caso recursivo: \verb|d[i][j] = d[i][j] or (d[i][k] and d[k][j]);|
		
		\subsubsection{Minimax}
		Dado un grafo con pesos, hallar el camino de $i$ hasta $j$ donde la arista más grande del camino sea lo más pequeña posible.\\
		Ejemplos: Que el peaje más caro sea lo más barato posible, que la autopista más larga sea lo más corta posible.\\ \quad \\
		Casos base:
		$\verb|d[i][j]| =
			\begin{cases}
				0         &\text{\quad si } i = j\\
				w_{i,j}   &\text{\quad si existe una arista entre $i$ y $j$}\\
				+\infty   &\text{\quad en otro caso}
			\end{cases}$ \\ \quad \\
		Caso recursivo: \verb|d[i][j] = min( d[i][j], max (d[i][k], d[k][j]) );|
		
		\subsubsection{Maximin}
		Dado un grafo con pesos, hallar el camino de $i$ hasta $j$ donde la arista más pequeña del camino sea lo más grande posible.\\
		Ejemplos: Que el trayecto menos seguro sea lo más seguro posible, que la autopista de menos carriles tenga la mayor cantidad de carriles.\\ \quad \\
		Casos base:
		$\verb|d[i][j]| =
			\begin{cases}
				+\infty   &\text{\quad si } i = j\\
				w_{i,j}   &\text{\quad si existe una arista entre $i$ y $j$}\\
				-\infty   &\text{\quad en otro caso}
			\end{cases}$ \\ \quad \\
		Caso recursivo: \verb|d[i][j] = max( d[i][j] , min(d[i][k], d[k][j]) )|
		
	
	\subsection{Algoritmo de Prim}
	Dado un grafo no dirigido y conexo, retorna el costo del árbol de mínima expansión de ese grafo.\\
	El costo del árbol de mínima expansión también se puede ver como el mínimo costo de las aristas de manera que haya un camino entre cualquier par de nodos.\\
	Complejidad: $O(m \operatorname{log} n)$ donde $n$ es el número de nodos y $m$ es el número de aristas.\\
		\source{./src/prim.cpp}
	
	\subsection{Algoritmo de Kruskal}
		\subsubsection{Union-Find}
		Union-Find es una estructura de datos para almacenar una colección conjuntos disjuntos (no tienen elementos en común) que cambian dinámicamente.\\
		Identifica en cada conjunto un ``padre'' que es un elemento al azar de ese conjunto y hace que todos los elementos del conjunto ``apunten'' hacia ese padre. \\
		Inicialmente se tiene una colección donde cada elemento es un conjunto unitario.\\
		Complejidad aproximada: $O(m)$ donde $m$ el número total de operaciones de initialize, union y join realizadas.\\
		\source{./src/union-find.cpp}
		
		\subsubsection{Algoritmo de Kruskal}
		Dado un grafo no dirigido y conexo, retorna el costo del árbol de mínima expansión de ese grafo.\\
		El costo del árbol de mínima expansión también se puede ver como el mínimo costo de las aristas de manera que haya un camino entre cualquier par de nodos.\\
		Utiliza Union-Find para ver rápidamente qué aristas generan ciclos.\\
		Complejidad: $O(m\operatorname{log} n)$ donde $n$ es el número de nodos y $m$ es el número de aristas.\\
		\source{./src/kruskal.cpp}
		
	\subsection{Algoritmo de máximo flujo}
		Dado un grafo con capacidades enteras, halla el máximo flujo entre una fuente $s$ y un sumidero $t$.\\
		Como el máximo flujo es igual al mínimo corte, halla también el mínimo costo de cortar aristas de manera que $s$ y $t$ queden desconectados.\\
		Si hay varias fuentes o varios sumideros poner una super-fuente / super-sumidero que se conecte a las fuentes / sumideros con capacidad infinita.\\
		Si los nodos también tienen capacidad, dividir cada nodo en dos nodos: uno al que lleguen todas las aristas y otro del que salgan todas las aristas y conectarlos con una arista que tenga la capacidad del nodo.\\
		Complejidad: $O(n \cdot m^2)$ donde $n$ es el número de nodos y $m$ es el número de aristas.\\
		\source{./src/maxflow.cpp}
	
\section{Teoría de números}
	\subsection{Divisores de un número}
	Imprime los divisores de un número (cuidado que no lo hace en orden).\\
	Complejidad: $O(\sqrt{n})$ donde $n$ es el número.\\
	\source{./src/divisors.cpp}
	
	\subsection{Máximo común divisor y mínimo común múltiplo}
	Para hallar el máximo común divisor entre dos números \verb|a| y \verb|b| ejecutar el comando \verb|__gcd(a, b)|.\\
	Para hallar el mínimo común múltiplo: \quad $ \displaystyle\operatorname{lcm}(a, b) = \frac{|a \cdot b|}{\operatorname{gcd}(a,b)}$
	
	\subsection{Criba de Eratóstenes}
	Encuentra los primos desde 1 hasta un límite $n$.\\
	Complejidad: $O(n)$ donde $n$ es el límite superior.\\
	\source{./src/sieve.cpp}
	
	\subsection{Factorización prima de un número}
	Halla la factorización prima de un número $a$ positivo. Si $a$ es negativo llamar el algoritmo con $|a|$ y agregarle -1 a la factorización.\\
	Se asume que ya se ha ejecutado el algoritmo para generar los primos hasta al menos $\sqrt{a}$.\\
	El algoritmo genera la lista de primos en orden de menor a mayor.\\
	Utiliza el hecho de que en la factorización prima de $a$ aparece máximo un primo mayor a $\sqrt{a}$.\\
	Complejidad aproximada: $O(\sqrt{a})$\\
	\source{./src/factorization.cpp}
	
	\subsection{Exponenciación logarítmica}
		\subsubsection{Propiedades de la operación módulo}
		\begin{itemize}
			\item $(a \operatorname{mod} n) \operatorname{mod} n = a \operatorname{mod} n$
			\item $(a + b) \operatorname{mod} n = ((a \operatorname{mod} n) + (b\operatorname{mod} n)) \operatorname{mod} n$
			\item $(a \cdot b) \operatorname{mod} n = ((a \operatorname{mod} n) \cdot (b\operatorname{mod} n)) \operatorname{mod} n$
			\item $\displaystyle\left(\frac{a}{b} \right) \operatorname{mod} n \neq \left(\frac{a \operatorname{mod} n}{b\operatorname{mod} n}\right) \operatorname{mod} n$
		\end{itemize}
		
		\subsubsection{Big mod}
		Halla rápidamente el valor de $b^{\,p} \operatorname{mod} m$ para $0 \leq b,p,m \leq 2147483647$\\
		Si se cambian los valores por \verb|long long| los límites se cambian por $0 \leq b,p \leq 9223372036854775807$ y $1 \leq m \leq 3037000499$.\\ 
		Complejidad: $O(\operatorname{log} p)$\\
		\source{./src/bigmod.cpp}
	
	\subsection{Combinatoria}
		\subsubsection{Coeficientes binomiales}
		Halla el valor de $\binom{n}{k}$ para $0 \leq k \leq n \leq 66$. Para $n > 66$ los valores comienzan a ser muy grandes y no caben en un \verb|long long|.\\
		Complejidad: $O(n^2)$\\
		\source{./src/binomial.cpp}
		
		\subsubsection{Propiedades de combinatoria}
		\begin{itemize}
			\item El número de permutaciones de $n$ elementos diferentes es $n!$
			\item El número de permutaciones de $n$ elementos donde hay $m_1$ elementos repetidos de tipo 1, $m_2$ elementos repetidos de tipo 2,~\ldots, $m_k$ elementos repetidos de tipo $k$ es $$\frac{n!}{m_1! m_2! \cdots m_k!} $$
			\item El número de permutaciones de $k$ elementos diferentes tomados de un conjunto de $n$ elementos es $$ \frac{n!}{(n-k)!} = k! \binom{n}{k}$$
		\end{itemize}
		
		
	
\section{Programación dinámica}
	\subsection{Problema de la mochila}
	Halla el valor máximo que se puede obtener al empacar un subconjunto de $n$ objetos en una mochila de tamaño $W$ cuando se conoce el valor y el tamaño de cada objeto.\\
	Complejidad: $O(n \times W)$ donde $n$ es el número de objetos y $W$ es la capacidad de la mochila.\\
	\source{./src/knapsack.cpp}
	
	
\section{Strings}
	\subsection{Longest common subsequence}
	Halla la longitud de la máxima subsecuencia (no substring) de dos cadenas $s$ y $t$.\\
	Una subsecuencia de una secuencia $s$ es una secuencia que se puede obtener de $s$ al borrarle algunos de sus elementos (probablemente todos) sin cambiar el orden de los elementos restantes.\\
	El algoritmo también se puede aplicar para vectores de elementos, no sólo para strings.\\
	Complejidad: $O(n \times m)$ donde $n$ es la longitud de $s$ y $m$ es la longitud de $t$.\\
	\source{./src/lcs.cpp}
	
	\subsection{Longest increasing subsequence}
	Halla la longitud de la subsecuencia creciente más larga que hay en un string~$s$ (también se puede usar con vectores).\\
	Complejidad: $O(n^2)$ donde $n$ es la longitud de~$s$.\\
	\source{./src/lis.cpp}
	
	\subsection{Algoritmo de KMP}
	Encuentra si el string \verb|needle| aparece en en string \verb|haystack|.\\
	Si no se retorna directamente \verb|true| cuando se halla la primera ocurrencia, el algoritmo encuentra todas las ocurrencias de \verb|needle| en \verb|haystack|.\\
	La primera parte del algoritmo llena el arreglo \verb|border| donde \verb|border[i]| es la longitud del borde del prefijo de \verb|needle| que termina en la posición \verb|i|. Un borde de una cadena $s$ es la cadena más larga que es a la vez prefijo y sufijo de $s$ pero que es diferente de $s$.\\
	Complejidad: $O(n)$ donde $n$ es el tamaño de \verb|haystack|.\\
	\source{./src/kmp.cpp}
	

\end{document}